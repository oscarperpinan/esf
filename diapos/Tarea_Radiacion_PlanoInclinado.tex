\documentclass[11pt]{article}
\usepackage[a4paper]{geometry}
\geometry{verbose,tmargin=2.5cm,bmargin=2.5cm,lmargin=2.5cm,rmargin=2.5cm}
\usepackage[utf8]{inputenc}
\usepackage[T1]{fontenc}
\usepackage{graphicx}
\usepackage{hyperref}
\hypersetup{
 pdfauthor={Oscar Perpiñán Lamigueiro},
 pdftitle={Actividad: cálculo de radiación},
 colorlinks=true,       % false: boxed links; true: colored links
 linkcolor=Brown,          % color of internal links
 citecolor=BrickRed,        % color of links to bibliography
 filecolor=black,      % color of file links
 urlcolor=Blue,           % color of external links 
 pdflang={Spanish}}
\usepackage[usenames,dvipsnames]{xcolor}
\usepackage[spanish]{babel}
\usepackage{mathpazo}
\usepackage{enumitem}

\author{Oscar Perpiñán Lamigueiro}
\date{}
\title{Actividad: cálculo de radiación en el plano de un generador\\\medskip
\large Energía Solar Fotovoltaica}
\begin{document}

\maketitle

En esta actividad vas a estimar las medias mensuales de la radiación
global incidente en el plano de tres generadores diferentes, todos
localizados en el hemisferio norte:
\begin{enumerate}
\item Un generador estático orientado al sur y con una inclinación de
  30º.
\item Un generador sobre un seguidor de eje horizontal Norte-Sur.
\item Un generador sobre un seguidor de doble eje.
\end{enumerate}

Seguiremos el siguiente itinerario para cada generador, teniendo en
cuenta que los puntos 1 a \ref{fin-procedimiento-horizontal} son
comunes a los tres generadores:

\begin{enumerate}
\item En la localización elegida en la tarea de cálculo de radiación
  horizontal, calcula la declinación, duración del día, e irradiación
  extra-atmosférica diaria en el plano horizontal \textbf{para los
    días promedio}.
\item Calcula el índice de claridad y fracción de difusa para las 12
  medias mensuales obtenidas en el plano horizontal, ya sea con la
  hibridación entre estaciones terrestres e imágenes de satélite, o
  únicamente con interpolación IDW. 
\item Con estos parámetros obtén las medias mensuales de irradiación
  difusa y directa diarias en el plano horizontal.
\item Calcula el coseno del ángulo cenital para cada día promedio
  (mes). Deberías obtener 24 valores por día (un total de 12·24=288
  valores por parámetro).
\item Calcula los perfiles intradiaarios rd y rg para cada dia
  promedio (mes). Deberías obtener 24 valores por mes (un total de
  12·24=288 valores por parámetro).
\item Obtén los perfiles de irradiancia difusa, global y diaria para
  cada día promedio (mes). Deberías obtener 24 valores por mes (un
  total de 12·24=288 valores por componente).\label{fin-procedimiento-horizontal}
\item Calcula el coseno del ángulo de incidencia para cada día
  promedio (mes). Deberías obtener 24 valores por día (un total de
  12·24=288 valores por parámetro).
\item Obtén la irradiancia en el plano del generador realizando la
  transformación de los valores de radiación en el plano horizontal al
  plano del generador.
\item Suma los resultados del paso anterior para obtener las 12 medias
  mensuales de radiación diaria (global, difusa y directa) en el plano
  del generador. Comprueba que estos resultados son superiores a los
  valores de la radiación en el plano horizontal, y que los valores
  con un seguidor de doble eje son superiores a los del eje
  horizontal, y estos superiores a los del generador estático.
\end{enumerate}


\end{document}