% Created 2015-02-04 mié 17:26
\documentclass[xcolor={usenames,svgnames,dvipsnames}]{beamer}
\usepackage[utf8]{inputenc}
\usepackage[T1]{fontenc}
\usepackage{fixltx2e}
\usepackage{graphicx}
\usepackage{longtable}
\usepackage{float}
\usepackage{wrapfig}
\usepackage{rotating}
\usepackage[normalem]{ulem}
\usepackage{amsmath}
\usepackage{textcomp}
\usepackage{marvosym}
\usepackage{wasysym}
\usepackage{amssymb}
\usepackage{hyperref}
\tolerance=1000
\usepackage{color}
\usepackage{listings}
\usepackage{mathpazo}
\usepackage{gensymb}
\usepackage{amsmath}
\bibliographystyle{plain}
\AtBeginSubsection[]{\begin{frame}[plain]\tableofcontents[currentsubsection,sectionstyle=show/shaded,subsectionstyle=show/shaded/hide]\end{frame}}
\AtBeginSection[]{\begin{frame}[plain]\tableofcontents[currentsection,hideallsubsections]\end{frame}}
\usepackage[emulate=units]{siunitx}
\sisetup{per=fraction, fraction=nice, decimalsymbol=comma}
\newunit{\wattpeak}{Wp}
\newunit{\watthour}{Wh}
\newunit{\amperehour}{Ah}
\hypersetup{colorlinks=true, linkcolor=Blue, urlcolor=Blue}
\setbeamercolor{alerted text}{fg=red!50!black} \setbeamerfont{alerted text}{series=\bfseries}
\usetheme[hideothersubsections]{Goettingen}
\usecolortheme{rose}
\usefonttheme{serif}
\author{Oscar Perpiñán Lamigueiro \\ \url{http://oscarperpinan.github.io}}
\date{}
\title{SFB: Diseño}
\hypersetup{
  pdfkeywords={},
  pdfsubject={},
  pdfcreator={Emacs 24.4.1 (Org mode 8.2.7c)}}
\begin{document}

\maketitle

\section{Caudal}
\label{sec-1}
\begin{frame}[label=sec-1-1]{Potencia hidráulica}
La \alert{potencia hidráulica}, $P_{H}$, necesaria para bombear agua es una
función de, la a*ltura vertical aparente*, $H_{v}$ y del \alert{caudal de
agua}, $Q$:$$P_{H}=g\cdot\rho\cdot Q\cdot H_{v}$$ donde g es la
aceleración de la gravedad, y $\rho$ es la densidad del agua.

Cambiando las unidades$$P_{H}=2.725\cdot Q\cdot H_{V}$$

con $P_{H}$ en watios, $H_{v}$ en metros y $Q$ en
$\si{\meter\cubed\per\hour}$.
\end{frame}

\begin{frame}[label=sec-1-2]{Potencia eléctrica de la motobomba}
Asumiendo que el agua bombeada sale por el conducto a una velocidad
insignificante, la potencia de salida de la bomba necesita satisfacer
$P_{H}$ más las \alert{perdidas de fricción en la tubería}, $P_{f}$.
Consecuentemente, la \alert{potencia eléctrica a la entrada de la motobomba},
$P_{el}$, es:$$P_{el}=\frac{P_{H}+P_{f}}{\eta_{mp}}$$ donde $\eta_{MP}$
es la \alert{eficiencia de la motobomba}.

El valor de $P_{H}+P_{f}$ es la \alert{potencia mecánica a la salida de la
bomba}. Este valor se asimila a una altura equivalente $H_{T}$ asociado
a un caudal determinado:$$H_{T}=H_{v}+H_{f}$$
\end{frame}

\begin{frame}[label=sec-1-3]{Potencia eléctrica del generador}
La *potencia eléctrica requerida por la motobomba es entregada por un
generador FV y un acondicionador de
potencia*:$$P_{el}=P_{g}^{*}\cdot\frac{G}{G^{*}}\frac{\eta_{g}}{\eta_{g}^{*}}\cdot\eta_{inv}$$

siendo $\eta_{inv}$ la \alert{eficiencia del equipo de acondicionamiento de
potencia}.
\end{frame}



\begin{frame}[label=sec-1-4]{Caudal diario}
El \alert{caudal diario} bombeado por este conjunto
es:$$Q_{d}=\intop_{d}\frac{P_{g}^{*}\cdot\frac{G}{G^{*}}\frac{\eta_{g}}{\eta_{g}^{*}}\cdot\eta_{inv}\cdot\eta_{mp}}{2.725\cdot H_{T}}\mathrm{dt}$$

Debido a las variaciones de la temperatura ambiente y de la irradiancia,
y también a causa del comportamiento dinámico de los pozos, \alert{todos los
parámetros mencionados anteriormente varían a lo largo del tiempo}. Por
tanto, la resolución de la anterior ecuación es tarea dificil.
\end{frame}

\begin{frame}[label=sec-1-5]{Necesidades de caudal}
\begin{itemize}
\item \alert{OMS}: 50 litros diarios por habitante.

\item En \alert{crisis humanitarias}, mínimo 3 litros diarios en climas templados
y 5 litros en climas cálidos.

\item En \alert{programas de cooperación}, 30 a 35 litros diarios por persona.

\item Para \alert{sistemas fotovoltaicos}, se recomienda 25 litros diarios por
habitante (fuentes comunitarias) o 45 litros (con grifo en cada
domicilio).

\item \alert{Contexto}: en grandes ciudades 250 litros diarios por habitante.
\end{itemize}
\end{frame}
\section{Altura}
\label{sec-2}
\begin{frame}[label=sec-2-1]{Altura constante}
\begin{itemize}
\item El supuesto de \alert{altura total de bombeo constante} sólo ocurre cuando,
por un lado, las \alert{pérdidas de fricción en la tubería son
despreciables} y, cuando por otro, el \alert{nivel del agua dentro del pozo
se mantiene constante}.

\item Lo primero se puede asegurar usando diámetros de tubería
suficientemente grandes: pérdidas de fricción por debajo del 5\% de la
altura total son un requisito de optimización (es decir,
$H_{f}<0.05\cdot H_{T}$).
\end{itemize}
\end{frame}

\begin{frame}[label=sec-2-2]{Altura total equivalente}
Se puede definir una \alert{altura total equivalente}, $H_{TE}$, como el
hipotético valor constante que llevaría al mismo volumen de agua
bombeada:

$$Q_{d}=\frac{P_{g}^{*}\cdot}{2.725\cdot G^{*}\cdot H_{TE}}\cdot\intop_{dia}G\cdot\frac{\eta_{g}}{\eta_{g}^{*}}\cdot\eta_{inv}\cdot\eta_{mp}\mathrm{dt}$$

Ahora, dada una $H_{TE}$, *la ecuación depende exclusivamente de las
condiciones meteorológicas y de las características de la bomba
fotovoltaica*. En la ecuación, $H_{OT}$ representa la altura desde la
salida de agua hasta el suelo.
\end{frame}

\begin{frame}[label=sec-2-3]{Caracterización de pozos}
Normalmente se realiza un \alert{ensayo de bombeo para caracterizar los
pozos}.

Éste consiste en extraer agua con una bomba portátil, y medir la caída
del nivel del agua en el pozo a un cierto caudal de bombeo y cuando
dicha caída se ha estabilizado.

Tres son los parámetros que completan la caracterización del pozo tras
el ensayo: el \alert{nivel estático}, $H_{st}$, el \alert{nivel dinámico}, $H_{dt}$,
y el \alert{caudal de ensayo}, $Q_{t}$.
\end{frame}

\begin{frame}[label=sec-2-4]{Caracterización de pozos}
Debe tomarse en consideración que la excesiva velocidad de extracción de
agua de un pozo puede dañar su superficie interna y provocar agujeros
que pueden llevar a un eventual colapso del pozo. Consiguientemente,
existe un \alert{caudal máximo para cada pozo}, $Q_{max}$ .

De hecho, la información de los ensayos mencionados de caracterización
de los pozos están, normalmente, referidos a este caudal máximo al que
se puede extraer el agua de ellos ($Q_{t}=Q_{max}$).
\end{frame}

\begin{frame}[label=sec-2-5]{Altura total equivalente}
*Es posible calcular *$H_{TE}$ mediante:

$$H_{TE}=H_{OT}+H_{ST}+(\frac{H_{DT}-H_{ST}}{Q_{T}})\cdot Q_{AP}+H_{f}(Q_{AP})$$

siendo $Q_{AP}$ el caudal aparente, calculado mediante
$Q_{AP}=\alpha\cdot Q_{d}$, y $\alpha=0.047\, h^{-1}$.
\end{frame}

\begin{frame}[label=sec-2-6]{Formula aproximada}
Si consideramos constantes a lo largo del tiempo las eficiencias del
generador fotovoltaico ($\dfrac{\eta_{g}}{\eta_{g}^{*}}=0.85$),
motobomba ($\eta_{mp}=0.35$) y variador ($\eta_{inv}=0.9$), es posible
\alert{calcular de forma aproximada la potencia nominal del generador}
necesaria para bombear un caudal diario $Q_{d}$ a una altura total
equivalente
$H_{TE}$:$$P_{g}^{*}=\frac{10\cdot H_{TE}\cdot Q_{d}}{G_{d}/G^{*}}$$

Por ejemplo, para bombear $\SI{30}{\meter\cubed\per\Day}$ a
$H_{TE}=\SI{40}{\meter}$ en un lugar de radiación diaria media
$G_{d}=\SI{5}{\kilowatthour\per\meter\squared\per\Day}$ se necesita un
generador fotovoltaico de:
$$P_{g}^{*}=\frac{10\cdot40\cdot30}{5}=\SI{2400}{\wattpeak}$$
\end{frame}

\section{Procedimiento de diseño}
\label{sec-3}


\begin{frame}[label=sec-3-1]{Procedimiento de diseño}
\begin{itemize}
\item A partir del caudal diario requerido y la altura total equivalente,
se calcula la potencia aproximada del generador FV.

\item Dividiendo el caudal diario requerido por la radiación diaria media,
se obtiene el caudal instantáneo medio.

\item Con este caudal, se acude al catálogo del fabricante (por ejemplo, la
nomenclatura de Grundfos para las bombas sumergibles es SP-XX-YY,
siendo XX el caudal instantáneo nominal de la bomba) y se elige un
grupo de bombas en el entorno.

\item Con los nomogramas se elige con el modelo concreto de bomba (caudal
nominal y número de etapas) y se obtiene un valor más preciso de la
potencia del generador. Este valor debe compararse con el inicial por
comprobación de errores.
\end{itemize}
\end{frame}

\begin{frame}[label=sec-3-2]{Procedimiento de diseño}
\includegraphics[width=.9\linewidth]{../figs/AbacoBomba.pdf}
\end{frame}

\begin{frame}[label=sec-3-3]{Procedimiento de diseño}
\begin{itemize}
\item Como seguridad, cuando la potencia entregada por el generador es
igual al 80\% de su potencia nominal, el caudal bombeado
correspondiente no debe exceder el máximo admisible por el pozo.

\item La tensión de entrada al variador debe
ser:$$V_{DC}=\frac{\sqrt{2}V_{AC}}{1.1}$$ luego para una bomba de
tensión de $230\, V_{ac}$ se necesita una tensión en la entrada que
no sea inferior a $V_{dc}\simeq300\, V_{dc}$. A partir de esta
tensión se configura el número de módulos por serie y el número de
ramas del generador.

\item A partir del caudal $Q_{AP}$ y de la longitud de tubería necesaria,
se elige el diámetro de la misma (en curvas del fabricante) de forma
que las pérdidas sean inferiores a un porcentaje prefijado de
$H_{te}$.
\end{itemize}
\end{frame}
% Emacs 24.4.1 (Org mode 8.2.7c)
\end{document}