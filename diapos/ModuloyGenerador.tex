% Created 2015-02-07 sáb 15:52
\documentclass[xcolor={usenames,svgnames,dvipsnames}]{beamer}
\usepackage[utf8]{inputenc}
\usepackage[T1]{fontenc}
\usepackage{fixltx2e}
\usepackage{graphicx}
\usepackage{longtable}
\usepackage{float}
\usepackage{wrapfig}
\usepackage{rotating}
\usepackage[normalem]{ulem}
\usepackage{amsmath}
\usepackage{textcomp}
\usepackage{marvosym}
\usepackage{wasysym}
\usepackage{amssymb}
\usepackage{hyperref}
\tolerance=1000
\usepackage{color}
\usepackage{listings}
\usepackage{mathpazo}
\usepackage{gensymb}
\usepackage{amsmath}
\bibliographystyle{plain}
\AtBeginSubsection[]{\begin{frame}[plain]\tableofcontents[currentsubsection,sectionstyle=show/shaded,subsectionstyle=show/shaded/hide]\end{frame}}
\AtBeginSection[]{\begin{frame}[plain]\tableofcontents[currentsection,hideallsubsections]\end{frame}}
\usepackage[emulate=units]{siunitx}
\sisetup{per=fraction, fraction=nice, decimalsymbol=comma}
\newunit{\wattpeak}{Wp}
\newunit{\watthour}{Wh}
\newunit{\amperehour}{Ah}
\hypersetup{colorlinks=true, linkcolor=Blue, urlcolor=Blue}
\setbeamercolor{alerted text}{fg=red!50!black} \setbeamerfont{alerted text}{series=\bfseries}
\usetheme[hideothersubsections]{Goettingen}
\usecolortheme{rose}
\usefonttheme{serif}
\author{Oscar Perpiñán Lamigueiro \\ \url{http://oscarperpinan.github.io}}
\date{}
\title{Energía Solar Fotovoltaica:\\Módulo y Generador}
\hypersetup{
  pdfkeywords={},
  pdfsubject={},
  pdfcreator={Emacs 24.4.1 (Org mode 8.2.7c)}}
\begin{document}

\maketitle

\section{Módulo Fotovoltaico}
\label{sec-1}

\subsection{Introducción}
\label{sec-1-1}

\begin{frame}[label=sec-1-1-1]{Módulo Fotovoltaico}
\begin{itemize}
\item Las características eléctricas de una célula no son suficientes para alimentar las cargas convencionales.

\item Es necesario realizar \alert{agrupaciones en serie y paralelo para entregar tensión y corriente adecuadas}.

\item Un \alert{módulo fotovoltaico} es una \alert{asociación de células} a las que \alert{protege de la intemperie}, las \alert{aisla eléctricamente} del exterior dando \alert{rigidez mecánica} al conjunto.

\item Existen multitud de módulos diferentes, tanto por su configuración eléctrica como por sus características estructurales y estéticas.
\end{itemize}
\end{frame}

\begin{frame}[label=sec-1-1-2]{Estructura de un módulo fotovoltaico}
\includegraphics[width=.9\linewidth]{../figs/panel_fv.png}

\begin{itemize}
\item La asociación de células es encapsulada en \alert{dos capas de EVA} (etileno-vinilo-acetato), entre una \alert{lámina frontal de vidrio} y una \alert{capa posterior} de un polímero termoplástico (frecuentemente se emplea el \alert{tedlar}).

\item Este conjunto es enmarcado en una \alert{estructura de aluminio anodizado} con el objetivo de aumentar la resistencia mecánica del conjunto y facilitar el anclaje del módulo a las estructuras de soporte.
\end{itemize}
\end{frame}

\begin{frame}[label=sec-1-1-3]{El vidrio frontal}
\begin{itemize}
\item Debe tener y mantener una \alert{alta transmisividad} en la banda espectral en la que trabajan las células solares.

\item Debe tener buena \alert{resistencia al impacto y a la abrasión}.

\item Su superficie debe ser de forma que combine un \alert{buen comportamiento antireflexivo} con la \alert{ausencia de bordes o desniveles} que faciliten
la acumulación de suciedad o dificulten la limpieza de ésta mediante la acción combinada del viento y la lluvia.

\item Frecuentemente se emplea \alert{vidrio templado con bajo contenido en hierro con algún tipo de tratamiento antireflexivo}.
\end{itemize}
\end{frame}

\begin{frame}[label=sec-1-1-4]{EVA}
\begin{itemize}
\item El \alert{encapsulante a base de EVA}, combinado con un tratamiento en vacío y las capas frontal y posterior, \alert{evita la entrada de humedad}
   en el módulo, señalada como la causa principal de la degradación a largo plazo de módulos fotovoltaicos.

\item Además, esta combinación permite obtener \alert{altos niveles de aislamiento eléctrico}.
\end{itemize}
\end{frame}

\begin{frame}[label=sec-1-1-5]{Configuración eléctrica}
\begin{itemize}
\item Una \alert{configuración eléctrica muy común} hasta hace unos años empleaba \alert{36 células en serie} para obtener módulos con potencias comprendidas
en el rango $\SIrange[range-phrase=-]{50}{100}{\wattpeak}$ con tensiones en MPP cercanas a los $\SI{15}{\volt}$ en funcionamiento.

\item Estos módulos eran particularmente adecuados para su acoplamiento con baterías de tensión nominal $\SI{12}{\volt}$ en los sistemas de
electrificación rural.

\item Con el protagonismo abrumador de los sistemas fotovoltaicos de conexión a red, esta configuración ha perdido importancia. Ahora son frecuentes los módulos de potencia superior a los $\SI{200}{\wattpeak}$ y tensiones en el rango $\SIrange[range-phrase=-]{30}{50}{\volt}$.
\end{itemize}
\end{frame}

\begin{frame}[label=sec-1-1-6]{Norma Internacional IEC 61215}
\begin{itemize}
\item Para los módulos compuestos por \alert{células de silicio cristalino} es de aplicación la \alert{norma internacional IEC-61215} \guillemotleft{}Crystalline Silicon
Terrestrial Photovoltaic (PV) Modules - Design Qualification and Type Approval\guillemotright{}.

\item Esta norma internacional recoge los \alert{requisitos de diseño y construcción} de módulos fotovoltaicos terrestres apropiados para su operación en períodos prolongados de tiempo bajo los efectos climáticos.

\item Detalla un \alert{procedimiento de pruebas} a los que se debe someter el módulo que desee contar con la certificación asociada a esta normativa
\end{itemize}
\end{frame}

\subsection{Modelado de un módulo}
\label{sec-1-2}

\begin{frame}[label=sec-1-2-1]{Suposiciones del modelo}
\begin{itemize}
\item Los efectos de la resistencia paralelo son despreciables

\item La corriente fotogenerada ($I_{L}$) es igual a la corriente de cortocircuito

\item En cualquier condición de operación $\exp(\frac{V+I\cdot R_{s}}{V_{t}})\gg1$
\end{itemize}

$$I=I_{sc}\cdot(1-\exp(\frac{V-V_{oc}+I\cdot R_{s}}{V_{t}})$$
\end{frame}

\begin{frame}[label=sec-1-2-2]{Efecto de la radiación y la temperatura}
\begin{itemize}
\item La \alert{corriente de cortocircuito} depende exclusivamente y de forma lineal de la \alert{irradiancia}.
\end{itemize}
$$I_{sc}=G_{ef}\cdot\frac{I_{sc}^{*}}{G^{*}}$$

\begin{itemize}
\item La* tensión de circuito abierto* depende exclusivamente de la \alert{temperatura de \emph{célula}}, y decrece linealmente con ella.
\end{itemize}
$$V_{oc}(T_{c})=V_{oc}^{*}+(T_{c}-T_{c}^{*})\cdot\frac{dV_{oc}}{dT_{c}}$$
\end{frame}

\begin{frame}[label=sec-1-2-3]{Temperatura de operación de célula}
\begin{itemize}
\item La \alert{temperatura de operación de la célula} depende de la \alert{temperatura y la irradiación}
\end{itemize}
$$T_{c}=T_{a}+G\cdot\frac{NOCT-20}{800}$$

\begin{itemize}
\item Como consecuencia, la \alert{eficiencia decrece} a razón de 0,5\% por grado centigrado.

\item La \alert{resistencia serie} es \alert{independiente} de las condiciones de operación.
\end{itemize}
\end{frame}

\begin{frame}[label=sec-1-2-4]{TONC}
\begin{itemize}
\item Temperatura que alcanza una \emph{célula} cuando su \emph{módulo} trabaja en las siguientes condiciones:

\begin{itemize}
\item Irradiancia: $G=\SI{800}{\watt\per\meter\squared}$

\item Espectro: el correspondiente a $AM=1.5$.

\item Incidencia normal

\item Temperatura \emph{ambiente}: $T_{a}=\SI{20}{\celsius}$.

\item Velocidad de viento: $v_{v}=\SI{1}{\meter\per\second}$.
\end{itemize}
\end{itemize}
\end{frame}

\begin{frame}[label=sec-1-2-5]{Ejemplo de cálculo}
Calcular el comportamiento eléctrico de un generador fotovoltaico constituido por 40 módulos, asociados en 4 ramas, bajo la suposición de factor de forma constante.

\begin{itemize}
\item Las condiciones de operación de este generador son:  $G_{ef}=700\, W/m^{2}$ y $T_{a}=34\celsius$.

\item De las fichas técnicas del módulo se extrae la siguiente información: $I_{sc}^{*}=3\, A$, $V_{oc}^{*}=19,8\, V$, $I_{mpp}^{*}=2,8\, A$ y $V_{mpp}^{*}=15.7\, V$.

\item Cada módulo está constituido por 33 células asociadas en serie. La TONC del módulo es de $43\celsius$.
\end{itemize}
\end{frame}

\subsection{Punto Caliente}
\label{sec-1-3}

\begin{frame}[label=sec-1-3-1]{Punto caliente}
\includegraphics[width=.9\linewidth]{../figs/AsociacionSerieCelulas.pdf}
\end{frame}

\begin{frame}[label=sec-1-3-2]{Punto caliente}
\includegraphics[width=.9\linewidth]{../figs/TensionCelula_Sombras.pdf}
\end{frame}

\begin{frame}[label=sec-1-3-3]{Punto caliente}
\includegraphics[width=.9\linewidth]{../figs/PotenciaCelula_Sombra.pdf}
\end{frame}

\begin{frame}[label=sec-1-3-4]{Diodo de paso}
\includegraphics[width=.9\linewidth]{../figs/AsociacionSerieCelulas_DiodosPaso.pdf}
\end{frame}

\begin{frame}[label=sec-1-3-5]{Curvas I-V con diodo de paso}
\includegraphics[width=.9\linewidth]{../figs/CurvaIV_DiodoPaso.pdf}
\end{frame}

\begin{frame}[label=sec-1-3-6]{Tensión con diodo de paso}
\includegraphics[width=.9\linewidth]{../figs/TensionesCelulasDiodos_DiodoPaso.pdf}
\end{frame}

\begin{frame}[label=sec-1-3-7]{Curvas Potencia con diodo de paso}
\includegraphics[width=.9\linewidth]{../figs/PotenciaCelulas_DiodoPaso.pdf}
\end{frame}

\begin{frame}[label=sec-1-3-8]{Curva Módulo con Diodos de Paso}
\includegraphics[width=.9\linewidth]{../figs/PotenciaModulo.pdf}
\end{frame}

\begin{frame}[label=sec-1-3-9]{Diodos de paso}
\includegraphics[width=.9\linewidth]{../figs/AsociacionSerieCelulas_DiodosPasoAlternos.pdf}
\end{frame}

\section{Generador Fotovoltaico}
\label{sec-2}

\subsection{Definición}
\label{sec-2-1}

\begin{frame}[label=sec-2-1-1]{Generador Fotovoltaico}
\begin{columns}
\begin{column}{7cm\textwidth}

\begin{itemize}
\item Un generador fotovoltaico es una asociación eléctrica de módulos fotovoltaicos para adaptarse a las condiciones de funcionamiento de una aplicación determinada.

\item Se compone de un total de $N_{p}\cdot N_{s}$ módulos, siendo $N_{p}$ el número de ramas y $N_{s}$ el número de módulos en cada serie.

\item El número de ramas define la corriente total del generador y el número de modulos por serie define la tensión del generador.
\end{itemize}
\end{column}


\begin{column}{5cm\textwidth}
\includegraphics[height=0.9\textheight]{../figs/AsociacionModulos.pdf}
\end{column}
\end{columns}
\end{frame}

\subsection{Pérdidas por dispersión}
\label{sec-2-2}

\begin{frame}[label=sec-2-2-1]{Pérdidas por dispersión}
\begin{block}{Definición del problema}
Los parámetros eléctricos de un módulo FV presentan dispersión: la producción energética será menor que la ideal.
\end{block}
\end{frame}

\begin{frame}[label=sec-2-2-2]{Distribución de valores de corriente y tensión}
La corriente de máxima potencia de un conjunto de módulos puede caracterizarse por una distribución tipo
Weibull$$f(I_{mpp})=\alpha\beta^{-\alpha}I_{mpp}^{\alpha-1}exp\left[-\left(\frac{I_{mpp}}{\beta}\right)^{\alpha}\right]$$
siendo $\alpha$ el factor de forma y $\beta$ el factor de escala de la distribución. La eficiencia de conexión serie
es:$$\eta_{cs}=\frac{I_{mpp}^{r}}{\overline{I_{mpp}}}$$ siendo $I_{mpp}^{r}$ la corriente de la rama, y $\overline{I_{mpp}}$ la media
de las corrientes del grupo de módulos.
\end{frame}

\begin{frame}[label=sec-2-2-3]{Eficiencia de conexión}
\begin{itemize}
\item A partir de la distribución y la definición de eficiencia de conexión serie puede deducirse que ésta se calcula mediante $$\eta_{cs}=N^{-\frac{1}{\alpha}}$$ siendo N el número de módulos en la serie. Por tanto, \alert{la eficiencia disminuye si aumenta N}. Asimismo, la eficiencia aumenta con $\alpha$.

\item Por otra parte, puede demostrarse que la \alert{tensión de un grupo de módulos} puede modelarse mediante una función \alert{gaussiana} y que \alert{la dispersión de valores de tensión es suficientemente baja para poder  considerar que la eficiencia de conexión de ramas en paralelo es igual a 1.}
\end{itemize}
\end{frame}

\begin{frame}[label=sec-2-2-4]{Clasificación de módulos}
\begin{itemize}
\item La dispersión de un conjunto depende inversamente del valor de
$\alpha$, así que un \alert{método para reducir las pérdidas por
dispersión} consiste en \alert{realizar clasificaciones} de los módulos
atendiendo a sus valores reales de corriente.

\item En sistemas de cierta entidad, puede ser conveniente realizar una
clasificación en tres categorías y crear cada rama con módulos de una
misma categoría.

\item Este método puede suponer reducciones del 2-3\% en las pérdidas
globales del sistema.
\end{itemize}
\end{frame}

\begin{frame}[label=sec-2-2-5]{Pérdidas por dispersión}
\begin{block}{Problema}
\begin{itemize}
\item Las clasificaciones se realizan en base a las médidas realizadas por
los fabricantes con"flash\guillemotright{}.

\item \alert{La indeterminación asociada a este método en relación a las medidas
a sol real son del mismo rango que la separación entre categorías.}
\end{itemize}
\end{block}
\end{frame}


\section{Ejemplos de generadores fotovoltaicos}
\label{sec-3}

\begin{frame}[label=sec-3-0-1]{}
\includegraphics[width=.9\linewidth]{../figs/Bifacial.jpg}
\end{frame}

\begin{frame}[label=sec-3-0-2]{}
\includegraphics[width=.9\linewidth]{../figs/er.jpg}
\end{frame}

\begin{frame}[label=sec-3-0-3]{}
\includegraphics[height=0.9\textheight]{../figs/TelefoniaRural.jpg}
\end{frame}

\begin{frame}[label=sec-3-0-4]{}
\includegraphics[width=.9\linewidth]{../figs/clasificacion2.jpg}
\end{frame}

\begin{frame}[label=sec-3-0-5]{}
\includegraphics[width=.9\linewidth]{../figs/dscf0997.jpg}
\end{frame}

\begin{frame}[label=sec-3-0-6]{}
\includegraphics[width=.9\linewidth]{../figs/ModulosRotos.jpg}
\end{frame}

\begin{frame}[label=sec-3-0-7]{}
\includegraphics[width=.9\linewidth]{../figs/p1010007.jpg}
\end{frame}

\begin{frame}[label=sec-3-0-8]{}
\includegraphics[width=.9\linewidth]{../figs/BarreraRuido.jpg}
\end{frame}
\begin{frame}[label=sec-3-0-9]{}
\includegraphics[width=.9\linewidth]{../figs/TorreguilInterior2.jpg}
\end{frame}

\begin{frame}[label=sec-3-0-10]{}
\includegraphics[width=.9\linewidth]{../figs/VistadesdeInterior.jpg}
\end{frame}
% Emacs 24.4.1 (Org mode 8.2.7c)
\end{document}