\documentclass[11pt]{article}
\usepackage[a4paper]{geometry}
\geometry{verbose,tmargin=2.5cm,bmargin=2.5cm,lmargin=2.5cm,rmargin=2.5cm}
\usepackage[utf8]{inputenc}
\usepackage[T1]{fontenc}
\usepackage{graphicx}
\usepackage{hyperref}
\hypersetup{
 pdfauthor={Oscar Perpiñán Lamigueiro},
 pdftitle={Actividad: cálculo de radiación},
 colorlinks=true,       % false: boxed links; true: colored links
 linkcolor=Brown,          % color of internal links
 citecolor=BrickRed,        % color of links to bibliography
 filecolor=black,      % color of file links
 urlcolor=Blue,           % color of external links 
 pdflang={Spanish}}
\usepackage[usenames,dvipsnames]{xcolor}
\usepackage[spanish]{babel}
\usepackage{mathpazo}
\usepackage{enumitem}

\author{Oscar Perpiñán Lamigueiro}
\date{}
\title{Actividad: cálculo de radiación\\\medskip
\large Energía Solar Fotovoltaica}
\begin{document}

\maketitle

En esta actividad vas a estimar las medias mensuales de la radiación
global incidente en el plano horizontal y en el plano de un generador.

\section{Radiación en el plano horizontal}

Para el cálculo de la \textbf{radiación en el plano horizontal}
seguiremos el siguiente itinerario:

\begin{enumerate}
\item Obtén series temporales de \textbf{medidas diarias} de radiación
  solar de 3 estaciones meteorológicas (recomendable longitud de 10
  años).
\item Filtra cada serie empleando límites físicos. \label{filtrado}
\item Elige una localización dentro del perímetro definido por las
  tres estaciones. Obtén las \textbf{12 medias mensuales} en esta
  localización empleando interpolación espacial (IDW) a partir de las
  \textbf{12 medias mensuales} de la radiación filtrada de cada
  estación (punto \ref{filtrado}). \label{idw}
\item Obtén medias temporales de radiación solar de un servicio
  satelital (preferiblemente CMSAF, empleando QGis o software similar;
  véanse los anexos \ref{sec:cmsaf} y \ref{sec:qgis}) para una región que
  cubra las tres estaciones\footnote{Los datos de radiación
    proporcionados por CMSAF son medias diarias de
    \textbf{irradiancia} (\(W/m^2\)). Por tanto, debes multiplicarlos
    por 24 para obtener valores diarios de
    \textbf{irradiación}.}.\label{satelite}
\item Compara las medias mensuales de las tres estaciones con las
  estimaciones satelitales empleando métricas estadísticas (MBD, RMSD,
  MAD).
\item Combina la estimación satelital en la localización del punto
  \ref{idw} con las medias mensuales obtenidas en ese punto.
\end{enumerate}

\clearpage

\section{Radiación en el plano del generador}

A continuación, realizaremos el cálculo de la \textbf{radiación en el
  plano del generador}. Este generador está localizado en el
hemisferio norte, orientado al sur y con una inclinación de
30º. Seguiremos el siguiente itinerario:

\begin{enumerate}
\item En la localización elegida en el punto \ref{idw} anterior,
  calcula la declinación, duración del día, e irradiación
  extra-atmosférica diaria en el plano horizontal \textbf{para los
    días promedio}.
\item Calcula el índice de claridad y fracción de difusa para las 12
  medias mensuales obtenidas en el punto \ref{satelite} del apartado
  anterior (o, en su defecto, punto \ref{idw}). Con estos parámetros
  obtén las medias mensuales de irradiación difusa y directa diarias
  en el plano horizontal.
\item Calcula el coseno del ángulo cenital y el coseno del ángulo de
  incidencia para cada día promedio (mes). Deberías obtener 24 valores
  por día (un total de 12·24=288 valores por parámetro).
\item Calcula los perfiles intradiaarios rd y rg para cada dia
  promedio (mes). Deberías obtener 24 valores por mes (un total de
  12·24=288 valores por parámetro).
\item Obtén los perfiles de irradiancia difusa, global y diaria para
  cada día promedio (mes). Deberías obtener 24 valores por mes (un
  total de 12·24=288 valores por componente).
\item Obtén la irradiancia en el plano del generador realizando la
  transformación de los valores del punto 12 del plano horizontal al
  plano del generador.
\item Suma los resultados del paso anterior para obtener las 12 medias
  mensuales de radiación diaria (global, difusa y directa) en el plano
  del generador. Comprueba que estos resultados son superiores a los
  valores de la radiación en el plano horizontal.
\end{enumerate}

\clearpage
\section{Anexo: CM-SAF}
\label{sec:cmsaf}
\href{https://wui.cmsaf.eu/safira/action/viewProduktSearch}{CM-SAF}
proporciona datos derivados de imágenes de satélite:
\begin{itemize}
\item \emph{Operational Products}: conjuntos de datos proporcionados casi en
  tiempo real.

\item \emph{Climate Data Records (CDR)}: series temporales de medidas de
  longitud, consistencia, y continuidad suficiente para determinar la
  variabilidad y cambios en el clima.
\end{itemize}

Los datos de radiación global se denominan \emph{Surface incoming
  shortwave radiation} (SIS). Para esta práctica utilizaremos
productos operacionales y las medias mensuales, disponibles
en:
\url{https://wui.cmsaf.eu/safira/action/viewProduktDetails?eid=21989_21990&fid=27}.

Para obtener estos datos para una región de interés hay que seguir
este procedimiento:
\begin{enumerate}
\item Elegir la región: \emph{Change projection / spatial resolution /
    domain}. En este apartado se puede definir la región de forma
  gráfica en el mapa o rellenando en los recuadros de longitud y
  latitud que hay a continuación. Una vez definida la zona, pasamos al
  siguiente apartado.
\item Definir el rango temporal: \emph{Proceed to time range
    selection}. En este apartado se indica el comienzo y final
  deseado. Al ser medias mensuales, obtendremos un fichero raster para
  cada mes de este rango. Por ejemplo, si elegimos un rango temporal
  que incluye 10 años, obtendremos 120 ficheros raster. Para esta
  práctica lo haremos de forma más simplificado, eligiendo únicamente
  un año para obtener 12 ficheros raster, por ejemplo desde 2022-01-01
  hasta 2022-12-31.
\item Una vez definida la región y el rango temporal, se puede
  realizar la solicitud: \emph{Add to order cart}. Para poder
  finalizar la solicitud es necesario tener usuario y contraseña. Si
  no es así, hay que registrarse de forma gratuita.
\item Tras realizar la solicitud se recibirá una notificación por
  correo electrónico pasado un tiempo que depende del tamaño de los
  ficheros solicitados. En este correo se incluye un enlace para
  realizar la descarga de los ficheros desde un servidor FTP.
\end{enumerate}

\section{Anexo: QGIS}
\label{sec:qgis}

\href{https://qgis.org/es/site/}{QGIS} is un proyecto de código
abierto de
un
\href{https://es.wikipedia.org/wiki/Sistema\_de\_informaci\%C3\%B3n\_geogr\%C3\%A1fica}{sistema
  de información geográfica} (SIG). Este software es capaz de manejar
datos vectoriales (puntos, líneas y polígonos) y tipo raster (matrices
de datos georeferenciados).

Deberías
leer
\href{https://docs.qgis.org/3.22/es/docs/user\_manual/working\_with\_raster/index.html}{este
  tutorial} para aprender a trabajar con ficheros tipo raster,
y
\href{http://docs.qgis.org/3.22/es/docs/user\_manual/working\_with\_vector/index.html}{este
  tutorial} para trabajar con datos vectoriales.

Hay varios plugins que proporcionan capacidades
adicionales. \href{https://docs.qgis.org/3.22/es/docs/training\_manual/qgis\_plugins/fetching\_plugins.html}{Este
  documento} es una guía en el proceso de instalación y utilización de
plugins. Te pueden interesar los siguientes:

\begin{itemize}
\item \href{https://plugins.qgis.org/plugins/pointsamplingtool/}{Point
    sampling tool}, para extraer valores de un fichero raster en un
  punto concreto.
\item
  \href{https://docs.qgis.org/3.22/es/docs/user\_manual/plugins/plugins\_interpolation.html}{Interpolation} plugin,
  para realizar una interpolación IDW de datos vectoriales.
\item Además, también es recomendable el plugin para
  utilizar \href{https://grass.osgeo.org/}{GRASS}. \href{http://docs.qgis.org/3.22/es/docs/user\_manual/grass\_integration/grass\_integration.html}{Este
    documento} proporciona instrucciones para instalarlo y trabajar
  con él.
\end{itemize}

\subsection*{Procedimiento}
\label{sec:orge72c353}

\begin{enumerate}
\item Añadir capa ráster: leer los 12 ficheros .nc de CMSAF.
\item Establecer proyección a EPSG:4326
\item Calculador raster: doble click en la capa, y en la expresión del
  cuadro inferior multiplicar por 24 (create on-the-fly raster).
\item Añadir capa de texto delimitado: leer fichero SIAR.csv
  (ubicaciones de estaciones). Establecer proyección a EPSG:4326.
\item Utilizar \emph{Point sampling tool}
  \begin{enumerate}
  \item \emph{Layer containing sampling points} : SIAR
  \item \emph{Layers with bands to get values from}: 12 SISmm.
  \item \emph{Output point vector layer}: \emph{Browse} y crear un
    nuevo fichero.
  \item Aceptar: aparece una nueva capa con el nombre del fichero
    creado.
  \item En esta nueva capa, con el botón derecho aparece un menú de
    opciones: tabla de atributos (valores extraídos). También se puede
    exportar.
  \end{enumerate}
\end{enumerate}


\end{document}