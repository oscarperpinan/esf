% Created 2023-02-26 dom 10:36
% Intended LaTeX compiler: pdflatex
\documentclass[aspectratio=169, usenames,svgnames,dvipsnames]{beamer}
\usepackage[utf8]{inputenc}
\usepackage[T1]{fontenc}
\usepackage{graphicx}
\usepackage{longtable}
\usepackage{wrapfig}
\usepackage{rotating}
\usepackage[normalem]{ulem}
\usepackage{amsmath}
\usepackage{amssymb}
\usepackage{capt-of}
\usepackage{hyperref}
\usepackage{color}
\usepackage{listings}
\usepackage{mathpazo}
\usepackage{gensymb}
\usepackage{amsmath}
\usepackage{diffcoeff}
\usepackage{steinmetz}
\usepackage{mathtools}
\usepackage{fancyvrb}
\DefineVerbatimEnvironment{verbatim}{Verbatim}{fontsize=\tiny, formatcom = {\color{black!70}}}
\bibliographystyle{plain}
\usepackage{siunitx}
\sisetup{output-decimal-marker={,}}
\DeclareSIUnit{\watthour}{Wh}
\DeclareSIUnit{\wattpeak}{Wp}
\DeclareSIUnit{\watthour}{Wh}
\DeclareSIUnit{\amperehour}{Ah}
\usepackage{steinmetz}
\hypersetup{colorlinks=true, linkcolor=Blue, urlcolor=Blue}
\usepackage[symbol, perpage]{footmisc}
\parskip=5pt
\usetheme{Boadilla}
\usecolortheme{rose}
\usefonttheme{serif}
\author{\href{https://oscarperpinan.github.io}{Oscar Perpiñán Lamigueiro}}
\date{}
\title{Energía Producida por un SFCR}
\subtitle{Energía Solar Fotovoltaica}
\institute[UPM]{Universidad Politécnica de Madrid}
\setbeamercolor{alerted text}{fg=blue!50!black} \setbeamerfont{alerted text}{series=\bfseries}
\AtBeginSubsection[]{\begin{frame}[plain]\tableofcontents[currentsubsection,sectionstyle=show/hide,subsectionstyle=show/shaded/hide]\end{frame}}
\AtBeginSection[]{\begin{frame}[plain]\tableofcontents[currentsection,hideallsubsections]\end{frame}}
\beamertemplatenavigationsymbolsempty
\setbeamertemplate{footline}[frame number]
\setbeamertemplate{itemize items}[triangle]
\setbeamertemplate{enumerate items}[circle]
\setbeamertemplate{section in toc}[circle]
\setbeamertemplate{subsection in toc}[circle]
\hypersetup{
 pdfauthor={\href{https://oscarperpinan.github.io}{Oscar Perpiñán Lamigueiro}},
 pdftitle={Energía Producida por un SFCR},
 pdfkeywords={},
 pdfsubject={},
 pdfcreator={Emacs 28.2 (Org mode 9.6)}, 
 pdflang={Spanish}}
\begin{document}

\maketitle

\section{Energía Producida por un SFCR}
\label{sec:org2891c78}
\begin{frame}[label={sec:org398ff81}]{Potencia en un SFCR}
\begin{itemize}
\item \alert{Potencia} a la Salida del Generador FV
\end{itemize}

$$P_{dc} = A_g \cdot \eta_g(G_{ef}, T_a) \cdot  G_{ef} = %
      \frac{\eta_g(G_{ef}, T_a)}{\eta_g^*} \cdot \frac{G_{ef}}{G^*} \cdot P_g^*$$

\begin{itemize}
\item \alert{Potencia} a la Salida del Inversor
\end{itemize}

$$P_{ac} = P_{dc} \cdot \eta_{inv}(P_{dc}, V_{dc}) =  P_{dc} \cdot \eta_{inv}(G_{ef}, T_a)$$

\begin{itemize}
\item \alert{Energía} Producida por un SFCR
\end{itemize}

$$E_{ac} = \int_T \frac{\eta_g(G_{ef}, T_a)}{\eta_g^*} \cdot
      \frac{G_{ef}}{G^*} \cdot \eta_{inv}(G_{ef}, T_a) \cdot P_g^*\quad \mathrm{dt}$$
\end{frame}

\begin{frame}[label={sec:org1d64535}]{Estimación de la Energía Producida}
$$E_{ac}=P_{g}^{*}\cdot\frac{G_{ef}}{G^*}\cdot PR\cdot (1-FS)$$

\begin{itemize}
\item \(E_{ac}\) es la \alert{energía producida} en un periodo.

\item \(G^*\) es la \alert{irradiancia} en condiciones estándar de medida (STC,
\(G_{stc}=\SI{1}{\kilo\watt\per\meter\squared}\),
\(T_c=\SI{25}{\celsius}\))

\item \(P_{g}^{*}\) es la \alert{potencia nominal} del generador FV
(\(\si{\kilo\wattpeak}\)) en STC

\item \(G_{ef}\) es la \alert{irradiación efectiva incidente} en el plano del
generador

\item \(PR\) es el \alert{rendimiento del sistema} o \emph{performance ratio}

\item \(FS\) es el \alert{factor de sombras}
\end{itemize}
\end{frame}

\begin{frame}[label={sec:org3fc65d5}]{Productividad}
En algunas ocasiones se habla de \alert{productividad} del sistema, \(Y_{f}\),
que es el cociente entre energía producida y potencia nominal del
\alert{generador}:
$$Y_{f}=\frac{E_{ac}}{P_{g}^{*}}\,(\si{\kilo\watthour\per\kilo\wattpeak})$$
\end{frame}

\begin{frame}[label={sec:orgb202828}]{Factor de sombras}
\begin{itemize}
\item \alert{El factor de sombras suele tomar valores alrededor del 2 al 4\%},
tanto en instalaciones estáticas como de seguimiento.

\item En casos específicos este factor puede ser más alto (por ejemplo,
debido a la existencia de edificios cercanos, o en aquellas plantas
con un nivel de ocupación de terreno superior al óptimo).
\end{itemize}
\end{frame}

\section{Performance Ratio}
\label{sec:org86892e3}

\begin{frame}[label={sec:org143bdc4}]{Definición}
\begin{itemize}
\item Está concebido para incluir todas las \alert{pérdidas que no tienen
dependencia con las condiciones meteorológicas}.

\item Este factor \emph{puede} caracterizar el funcionamiento de un sistema
\alert{independientemente de la localidad}.

\item En sentido estricto no es cierto porque sí hay relación con la
meteorología del lugar.

\item Sin embargo, dado que estos factores son de segundo orden comparados
con la relación entre potencia e irradiancia, \alert{suele aceptarse} que el
\alert{PR} sirve para caracterizar la \alert{calidad de un sistema fotovoltaico}.
\end{itemize}
\end{frame}


\begin{frame}[label={sec:org48aa246}]{Desglose de pérdidas}
\begin{itemize}
\item \alert{Dispersión de parámetros} entre los módulos que componen el
generador (2-4\%)

\item \alert{Tolerancia de potencia} de los módulos respecto a sus
características nominales (3\%)

\item \alert{Temperatura} de funcionamiento de los módulos (5-8\%)

\item Conversión DC/AC realizada por el \alert{inversor} (8-12\%)

\item \alert{Efecto Joule} en los cables (2-3\%)

\item Conversión BT/MT realizada por el \alert{transformador} (2-3\%)

\item \alert{Disponibilidad} del sistema (0,5-1\%)
\end{itemize}
\end{frame}


\begin{frame}[label={sec:org155f0c8}]{Valores reales}
\begin{itemize}
\item El análisis de funcionamiento de diversos sistemas FV europeos ha
mostrado que el rango de valores que toma el \emph{performance ratio} es
bastante amplio, con mínimos de 0,4 y máximos de 0,85.

\item Para sistemas instalados entre 1980 a 1990, \alert{el valor promedio ha sido de
0,7}.

\item Para sistemas instalados entre 2005 a 2012, \alert{el valor promedio ha
sido de 0,8}.
\end{itemize}
\end{frame}
\end{document}