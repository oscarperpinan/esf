% Created 2023-01-23 lun 16:27
% Intended LaTeX compiler: pdflatex
\documentclass[aspectratio=169, usenames,svgnames,dvipsnames]{beamer}
\usepackage[utf8]{inputenc}
\usepackage[T1]{fontenc}
\usepackage{graphicx}
\usepackage{longtable}
\usepackage{wrapfig}
\usepackage{rotating}
\usepackage[normalem]{ulem}
\usepackage{amsmath}
\usepackage{amssymb}
\usepackage{capt-of}
\usepackage{hyperref}
\usepackage{color}
\usepackage{listings}
\usepackage{mathpazo}
\usepackage{gensymb}
\usepackage{amsmath}
\usepackage{diffcoeff}
\usepackage{steinmetz}
\usepackage{mathtools}
\usepackage{fancyvrb}
\DefineVerbatimEnvironment{verbatim}{Verbatim}{fontsize=\tiny, formatcom = {\color{black!70}}}
\bibliographystyle{plain}
\usepackage{siunitx}
\sisetup{output-decimal-marker={,}}
\DeclareSIUnit{\watthour}{Wh}
\DeclareSIUnit{\wattpeak}{Wp}
\DeclareSIUnit{\watthour}{Wh}
\DeclareSIUnit{\amperehour}{Ah}
\usepackage{steinmetz}
\hypersetup{colorlinks=true, linkcolor=Blue, urlcolor=Blue}
\renewcommand{\thefootnote}{\fnsymbol{footnote}}
\parskip=5pt
\usetheme{Boadilla}
\usecolortheme{rose}
\usefonttheme{serif}
\author{\href{https://oscarperpinan.github.io}{Oscar Perpiñán Lamigueiro}}
\date{}
\title{Bases de Datos de Radiación Solar}
\subtitle{Energía Solar Fotovoltaica}
\institute[UPM]{Universidad Politécnica de Madrid}
\setbeamercolor{alerted text}{fg=blue!50!black} \setbeamerfont{alerted text}{series=\bfseries}
\AtBeginSubsection[]{\begin{frame}[plain]\tableofcontents[currentsubsection,sectionstyle=show/hide,subsectionstyle=show/shaded/hide]\end{frame}}
\AtBeginSection[]{\begin{frame}[plain]\tableofcontents[currentsection,hideallsubsections]\end{frame}}
\beamertemplatenavigationsymbolsempty
\setbeamertemplate{footline}[frame number]
\setbeamertemplate{itemize items}[triangle]
\setbeamertemplate{enumerate items}[circle]
\setbeamertemplate{section in toc}[circle]
\setbeamertemplate{subsection in toc}[circle]
\hypersetup{
 pdfauthor={\href{https://oscarperpinan.github.io}{Oscar Perpiñán Lamigueiro}},
 pdftitle={Bases de Datos de Radiación Solar},
 pdfkeywords={},
 pdfsubject={},
 pdfcreator={Emacs 28.2 (Org mode 9.6)}, 
 pdflang={Spanish}}
\begin{document}

\maketitle

\section{Introducción}
\label{sec:orgf8aec55}

\begin{frame}[label={sec:org9ad897a}]{Radiación Solar y Sistemas Fotovoltaicos}
\begin{itemize}
\item La \alert{energía producida} por un sistema fotovoltaico depende principalmente de la \alert{radiación incidente} en el generador.

\item Consecuentemente, la \alert{estimación del comportamiento} de un sistema FV en un determinado lugar durante un período temporal exige \alert{conocer la radiación solar disponible en el plano del generador}.
\end{itemize}

\begin{center}
\includegraphics[width=.9\linewidth]{../figs/GCPVScheme.pdf}
\end{center}
\end{frame}

\begin{frame}[label={sec:org97fe9bb}]{La radiación solar no se puede calcular analíticamente}
\begin{itemize}
\item La radiación solar que alcanza la superficie terrestre es el resultado de complejas interacciones en la atmósfera.
\item Para estimar la radiación se necesitan medidas terrestres o imágenes de satélite.
\end{itemize}
\begin{center}
\includegraphics[height=0.5\textheight]{../figs/SolarRadiationComponents_NREL.png}
\end{center}
\end{frame}

\begin{frame}[label={sec:org0565527}]{Ángulo de Inclinación}
\begin{itemize}
\item Los generadores FV tienen un \alert{ángulo de inclinación positivo} para maximizar el rendimiento.
\item Este ángulo depende de la \alert{latitud} del lugar y de la \alert{aplicación del sistema}.
\end{itemize}

\begin{center}
\includegraphics[height=0.5\textheight]{../figs/PVUrban.png}
\end{center}
\end{frame}

\begin{frame}[label={sec:org13f7063}]{Bases de Datos de Radiación Solar}
\begin{itemize}
\item Por tanto, es inviable mantener una base de datos de radiación solar \alert{incidente}.
\item Las \alert{bases de datos} registran radiación en el \alert{plano horizontal}.
\item La estimación de la radiación incidente en el plano inclinado requiere un \alert{procedimiento de transposición}.
\end{itemize}
\end{frame}


\begin{frame}[label={sec:orgca6a070}]{Variabilidad Temporal y Espacial}
\begin{itemize}
\item La irradiancia solar extraterrestre depende de la latitud y el instante temporal (\emph{proceso determinista}).
\item La irradiancia solar incidente en la superficie terrestre es resultado de la interacción con la atmósfera cambiante: \alert{variabilidad temporal y espacial} (\emph{proceso estocástico}).
\end{itemize}
\end{frame}

\begin{frame}[label={sec:org8ac2eee}]{Variabilidad Temporal}
Variabilidad de la irradiación diaria, mensual y anual durante el período comprendido entre 2001-2008 en Carmona, Sevilla
\begin{center}
\includegraphics[width=.9\linewidth]{../figs/VariabilidadRadiacionDiario.pdf}
\end{center}

\nocite{Perpinan2009}
\end{frame}

\begin{frame}[label={sec:orgbe8a570}]{Variabilidad Temporal}
\[
\sigma_{\overline{G}}=\frac{\sigma_{G}}{\sqrt{N}}
\]

\begin{itemize}
\item Predicción para un (día, mes, año) \alert{determinado}: 

\begin{itemize}
\item Intervalo de confianza del 95\% acotado por \(1.96\cdot\sigma_{G}\)
\end{itemize}

\item Predicción para un (día, mes, año) \alert{promedio (durante N años)}: 

\begin{itemize}
\item Intervalo de confianza del 95\% acotado por \(1.96\cdot\sigma_{\overline{G}}\)
\end{itemize}
\end{itemize}
\end{frame}

\begin{frame}[label={sec:orgfa17d63}]{Variabilidad Espacial}
\begin{center}
\includegraphics[width=0.9\textwidth]{../figs/SpatialVariability.jpg}
\end{center}

\[
COV = 1/G_p \sqrt{\frac{\sum_1^{n}(G_p^2 - G_i^2)}{n}}
\]

\nocite{Gueymard.Wilcox2011a}
\end{frame}

\begin{frame}[label={sec:orgc4d559f}]{Variabilidad Espacial}
\begin{center}
\begin{center}
\includegraphics[height=0.9\textheight]{../figs/SpatialVariability_Annual.jpg}
\end{center}
\end{center}
\end{frame}

\begin{frame}[label={sec:orgdbd1907}]{Estimación a partir de Medidas}
\begin{itemize}
\item Para estimar la radiación incidente es necesario contar con:
\begin{itemize}
\item \alert{Medidas cercanas} (variabilidad espacial): distancia no superior a 10 km.
\item \alert{Series temporales} largas (variabilidad temporal): 10 años.
\end{itemize}
\end{itemize}
\end{frame}

\begin{frame}[label={sec:org3d0e65f}]{Fuentes de datos}
\begin{itemize}
\item \alert{Estaciones meteorológicas}
\begin{itemize}
\item Series largas y con tiempos de muestreo altos.
\item Baja resolución espacial (medidas puntuales)
\item Precisión en caso de medida directa.
\item Tipos: 
\begin{itemize}
\item Con medidor de radiación
\item Sin medidor de radiación (modelos empíricos).
\end{itemize}
\end{itemize}
\end{itemize}

\pause

\begin{itemize}
\item \alert{Imágenes de satélite}

\begin{itemize}
\item Tiempos de muestreo bajos (mejorando)

\item Resolución espacial alta

\item Error debido a la estimación.
\end{itemize}
\end{itemize}

\pause 

\begin{itemize}
\item \alert{Híbrido}

\begin{itemize}
\item Medidas terrestres combinadas con imágenes de satélite
\end{itemize}
\end{itemize}
\end{frame}

\section{Estaciones Meteorológicas}
\label{sec:orgad847cf}

\subsection{Fundamentos}
\label{sec:org1d7ea62}
\begin{frame}[label={sec:org3dbba30}]{Medida directa}
La medida directa de radiación solar se realiza con un piranómetro.
\begin{columns}
\begin{column}{0.4\columnwidth}
\begin{center}
\begin{center}
\includegraphics[width=0.8\textwidth]{../figs/piranometro.jpg}
\end{center}
\end{center}
\end{column}
\begin{column}{0.6\columnwidth}
\begin{itemize}
\item Pila termoeléctrica (termopares con barniz negro)
\begin{itemize}
\item Alojamiento con dos hemiesferas de cristal.
\item Flujo de calor por radiación provoca tensión eléctrica en termopila.
\end{itemize}
\item Respuesta espectral plana para radiación visible.
\item Respuesta perfecta al coseno del ángulo de incidencia (pérdidas por reflexión).
\end{itemize}
\end{column}
\end{columns}
\end{frame}

\begin{frame}[label={sec:org68f528c}]{Medida directa}
\begin{block}{La red de estaciones que miden directamente radiación es escasa para estimaciones precisas en regiones grandes}
\begin{itemize}
\item Un piranómetro requiere mantenimiento y calibración frecuente.
\item La proporción de estaciones con piranómetros es baja respecto a
las que miden temperatura ambiente y precipitación (1:500).
\end{itemize}
\end{block}
\end{frame}

\begin{frame}[label={sec:org020c069}]{Modelos empíricos}
Frente a la baja densidad de estaciones con medida directa de radiación se emplean modelos empíricos

\begin{itemize}
\item Relaciones entre radiación y otras variables
\begin{itemize}
\item Horas de brillo (\emph{sunshine duration})
\item Cobertura nubosa
\item Temperatura ambiente
\item Precipitación
\item Humedad
\item \ldots{}
\end{itemize}
\item Los coeficientes de los modelos sólo se pueden ajustar en estaciones
con medidas de radiación.
\item Los coeficientes dependen del lugar de ajuste, pero se pueden
interpolar para otras localizaciones.
\end{itemize}
\end{frame}

\begin{frame}[label={sec:org14ceac6}]{Ejemplos de modelos empíricos}
\begin{itemize}
\item Radiación y Horas de Brillo (Angstrom y Prescott)
\end{itemize}

\[
\frac{G(0)}{B_o(0)} = a_1 + b_1 \frac{S}{S_o}
\]

\begin{itemize}
\item Radiación y Temperatura (Bristow y Campbell)
\end{itemize}
\[
G(0) = a \left(1 - \exp(-b \Delta T^c)\right) \cdot B_o(0)
\]

\begin{itemize}
\item Variaciones con más variables: Lluvia (si/no), rango antes y después, velocidad viento, humedad relativa.
\end{itemize}

\[
  G(0) = a \left(1 - \exp(-b \Delta T^c)\right) \cdot B_o(0) \cdot \left(1 +
    \sum_1^n p_j \cdot v_j \right) + p_{n+1}
\]

\nocite{Antonanzas-Torres.Sanz-Garcia.ea2013}
\end{frame}


\subsection{Fuentes de Datos}
\label{sec:org441793a}

\begin{frame}[label={sec:orgdb2339b}]{Wiki con recursos}
\begin{block}{}
\url{https://github.com/oscarperpinan/mds/wiki}
\end{block}
\end{frame}


\begin{frame}[label={sec:org8befad3}]{Baseline Surface Radiation Network}
\begin{block}{\url{http://www.bsrn.awi.de/}}
\begin{itemize}
\item BSRN proporciona datos casi continuos, a largo plazo, observados in situ, de la superficie terrestre e irradiancias de banda ancha (infrarrojo solar y térmico) de una red de más de 50 sitios globalmente diversos.

\item Se emplea para la validación y confirmación de modelos satelitales y otros.
\end{itemize}

\begin{center}
\begin{center}
\includegraphics[height=0.5\textheight]{../figs/BSRN.png}
\end{center}
\end{center}
\end{block}
\end{frame}

\begin{frame}[label={sec:orgbd764d8}]{Measurement and Instrumentation Data Center NREL}
Radiación global, directa y difusa (y otras variables) con muestreo de
  1 min en diversas localidades de EEUU.

\url{http://www.nrel.gov/midc/}

\begin{center}
\begin{center}
\includegraphics[height=0.5\textheight]{../figs/NRELStation.jpg}
\end{center}
\end{center}
\end{frame}



\begin{frame}[label={sec:org9cdb529}]{SIAR}
\begin{block}{\url{https://eportal.mapa.gob.es/websiar/Inicio.aspx}}
\begin{itemize}
\item El Sistema de Información Agroclimática para el Regadío (SiAR)
registra datos agroclimáticos relacionados con demanda hídrica de
las zonas de riego.

\item Más de 400 estaciones.

\item Valores diarios y horarios
\end{itemize}

\begin{center}
\begin{center}
\includegraphics[height=0.35\textheight]{../figs/EstacionesSIAR.jpeg}
\end{center}
\end{center}
\end{block}
\end{frame}

\begin{frame}[label={sec:org49fba02}]{SIAR}
\begin{block}{Sensores}
\begin{itemize}
\item Temperatura y Humedad
\item Piranómetro
\item Anemoveleta
\item Pluviómetro
\item Temperatura del suelo  (algunas)
\end{itemize}

\begin{center}
\begin{center}
\includegraphics[height=0.4\textheight]{../figs/EstacionSIAR.png}
\end{center}
\end{center}
\end{block}
\end{frame}


\begin{frame}[label={sec:org8b6a0d9}]{AEMET}
\begin{block}{\href{http://www.aemet.es/es/eltiempo/observacion/radiacion}{Radiación}}
\begin{itemize}
\item Alrededor de 30 estaciones en todo el territorio.
\item Medidas de global, difusa y directa.
\item Sólo gráficas.
\end{itemize}
\end{block}

\begin{block}{\href{http://www.aemet.es/es/eltiempo/observacion/ultimosdatos}{Estaciones \guillemotleft{}convencionales\guillemotright{}}}
\begin{itemize}
\item Presión, temperatura, viento, humedad, lluvia.
\item Permite descarga de datos horarios por día.
\end{itemize}
\end{block}
\end{frame}

\begin{frame}[label={sec:org1f21c4e}]{Redes de Comunidades Autónomas}
\begin{itemize}
\item \href{https://www.meteogalicia.gal/observacion/estacions/estacions.action?request\_locale=es}{Meteogalicia}
\item \href{http://meteo.navarra.es/estaciones/mapadeestaciones.cfm}{MeteoNavarra}
\item \href{http://www.meteo.cat/observacions/xema}{Cataluña}
\item \href{http://www.euskalmet.euskadi.net/s07-5853x/es/meteorologia/lectur.apl?e\%3D5}{MeteoEuskadi}
\item \href{http://www.juntadeandalucia.es/medioambiente/servtc5/WebClima/?lr\%3Dlang\_es}{Andalucía}
\end{itemize}
\end{frame}


\section{Imágenes de Satélite}
\label{sec:orgeec113c}

\subsection{Fundamentos}
\label{sec:org86f0a1a}

\begin{frame}[label={sec:orga94185e}]{Fundamentos}
\begin{itemize}
\item Los satélites meteorológicos están equipados con \alert{radiómetros}
(sensores de radiación electromagnética a diferentes frecuencias)
que captan \alert{radiación emitida por la Tierra}.

\item La radiación emitida por la Tierra depende de la \alert{reflexión del
suelo}, y la \alert{geometría y composición de la atmósfera}.

\item Diferentes fenómenos físicos se detectan en \alert{bandas de frecuencias}
distintas (canales).

\item Existen diversos procedimientos para \alert{estimar radiación solar} en
superficie a partir de la información de los diferentes canales del
radiómetro.
\end{itemize}
\end{frame}

\begin{frame}[label={sec:org5a90dc5}]{Satelites Geoestacionarios Europeos: Meteosat}
\begin{itemize}
\item \alert{MFG}: Meteosat First Generation (7 satélites)
\begin{itemize}
\item Equipados con el radiómetro MVIRI (Meteosat Visible and Infrared Imager).
\item Tres canales: visible, infrarrojo, vapor de agua.
\end{itemize}
\item \alert{MSG}: Meteosat Second Generation (4 satélites)
\begin{itemize}
\item Equipados con dos radiómetros:
\begin{itemize}
\item \alert{SEVIRI} (Spinning Enhanced Visible and InfraRed Imager): 12 canales
\item GERB (Geostationary Earth Radiation Budget): infrarrojo visible.
\end{itemize}
\end{itemize}
\item \alert{MTG}: Meteosat Third Generation (1 satélite, por ahora)
\end{itemize}
\begin{center}
\begin{center}
\includegraphics[height=0.3\textwidth]{../figs/Tierra_MSG.jpg}
\end{center}
\end{center}
\end{frame}


\begin{frame}[label={sec:orgd5c50f5}]{Procedimientos: Heliosat-2}
\begin{block}{Pasos}
\begin{itemize}
\item Establecer \alert{albedo de referencia} (\emph{suelo}).
\item Estimar \alert{índice de cobertura nubosa}.
\item Estimar radiación en superficie a partir de cobertura nubosa y \alert{modelo de cielo claro}.
\end{itemize}
\end{block}

\begin{block}{}
\begin{itemize}
\item Empleado para base HelioClim
\item Usan datos de SEVIRI
\item Accesible via SoDa: \url{https://www.soda-pro.com/help/helioclim/heliosat-2}
\end{itemize}

\nocite{Rigollier.Lefevre.ea2004}
\end{block}
\end{frame}

\begin{frame}[label={sec:orge660b6d}]{Procedimientos: CM SAF}
\alert{Fundamento}
\begin{itemize}
\item Se emplea el modelo libRadtran (\alert{Radiative Transfer Model, RTM)}, para
generar una matriz de estados (\alert{Look-up table, LUT}) que relaciona la
transmitancia atmosférica y el albedo de la atmósfera para
variedad de estados.
\item La irradiancia en superficie se estima multiplicando la
irradiancia extra-atmosférica por la \alert{transmitancia atmosférica
determinada interpolando en la LUT}.
\end{itemize}
\end{frame}

\begin{frame}[label={sec:orgdeb8fd7}]{Procedimientos: CM SAF}
\begin{itemize}
\item \alert{Dos LUTs}: cielo nuboso, cielo claro.
\begin{itemize}
\item \alert{Cielo nuboso}:
\begin{itemize}
\item Estimación de albedo y estado atmosférico a partir de imágenes.
\item Estimación de transmitancia interpolando en LUT para cielo nuboso.
\end{itemize}
\item \alert{Cielo claro}:
\begin{itemize}
\item Estimación de transmitancia interpolando en LUT para cielo claro \alert{sin estimación previa} de albedo.
\end{itemize}
\end{itemize}

\item Emplean datos del \alert{radiómetro MSG/SEVIRI}
\end{itemize}

\nocite{Mueller.Matsoukas.ea2009}
\end{frame}



\begin{frame}[label={sec:orgc01593b}]{Procedimientos: LSA SAF}
\begin{itemize}
\item Generación de \alert{máscara de nubes} a partir de imagen usando algoritmo de \href{http://www.nwcsaf.org/}{NWC-SAF}.
\item Para \alert{zonas sin nubes}: modelo de cielo claro sin usar datos de imagen.
\item Para \alert{zonas cubiertas}: modelo de transmitancia atmosférica a partir de imágenes.
\item Emplean datos del \alert{radiómetro MSG/SEVIRI}
\end{itemize}

\nocite{Geiger.Meurey.ea2008}
\end{frame}


\subsection{Fuentes de Datos}
\label{sec:orgfea60cd}

\begin{frame}[label={sec:org8159019}]{Wiki con recursos}
\begin{block}{}
\url{https://github.com/oscarperpinan/mds/wiki}
\end{block}
\end{frame}


\begin{frame}[label={sec:org6fd68f2}]{SSE-NASA}
\begin{block}{Surface meteorology and Solar Energy (SSE)}
\begin{itemize}
\item 200 parámetros meteorológicos y de energía solar derivados de imágenes de satélite.
\item Base de datos de casi 40 años.
\item Resolución 1ºx1º
\end{itemize}

\url{https://power.larc.nasa.gov/}
\end{block}
\end{frame}

\begin{frame}[label={sec:org51b0ab6}]{EUMETSAT - SAF}
\begin{itemize}
\item \alert{\href{http://www.eumetsat.int}{EUMETSAT}} es la agencia europea de satélites en operación, para la monitorización de la meteorología, clima y el medio ambiente.
\item \alert{\href{https://www.eumetsat.int/about-us/satellite-application-facilities-safs}{Satellite Application Facilities} (SAFs)}
\begin{itemize}
\item Centros dedicados al procesamiento de datos de satélite.
\item Generan y distribuyen los productos y servicios EUMETSAT.
\end{itemize}
\end{itemize}
\end{frame}

\begin{frame}[label={sec:org2fab4e5}]{SAFs}
\begin{itemize}
\item \href{https://wui.cmsaf.eu/safira/action/viewProduktSearch}{SAF on Climate Monitoring (CM SAF)}: datos derivados de imágenes de satélite adecuados para la monitorización del clima.

\begin{itemize}
\item Operational Products: conjuntos de datos proporcionados casi en tiempo real.

\item Climate Data Records (CDR): series temporales de medidas de longitud, consistencia, y continuidad suficiente para determinar la variabilidad y cambios en el clima.
\end{itemize}

\item \href{https://landsaf.ipma.pt/en/}{SAF on Land Surface Analysis} (LSA SAF): genera, archiva y distribuye productos operacionales con un conjunto de parámetros relacionados con la radiación en superficie, la evotranspiración, cobertura vegetal e incendios.
\end{itemize}
\end{frame}

\begin{frame}[label={sec:orgeb522de}]{SAFs: Radiación}
\begin{itemize}
\item \alert{CM SAF}: Surface incoming shortwave radiation (\href{https://wui.cmsaf.eu/safira/action/viewProduktDetails?eid=21987\_21988\&fid=27}{SIS})

\begin{itemize}
\item AEMET ha analizado las estimaciones para España en su \href{http://www.aemet.es/es/serviciosclimaticos/datosclimatologicos/atlas\_radiacion\_solar}{Atlas de Radiación}.
\end{itemize}

\item \alert{LSA SAF}: Down-welling surface short-wave radiation flux (\href{https://landsaf.ipma.pt/en/products/longwave-shortwave-radiation/dssf/}{DSSF})
\end{itemize}
\end{frame}

\begin{frame}[label={sec:orgfa54d94}]{ADRASE - CIEMAT}
\begin{block}{\url{http://adrase.es}}
\begin{itemize}
\item Radiación solar media mensual, resolución aproximada de 5x5 km.
\begin{itemize}
\item Media mensual y anual más probable durante un periodo de largo
plazo (imágenes de satélite, modelo aproximadamente Heliosat)
\item Variabilidad esperada de los valores diarios mensuales: (series
largas de datos de estaciones de AEMET y extrapolación espacial
con IDW)
\end{itemize}
\end{itemize}

\begin{center}
\begin{center}
\includegraphics[height=0.35\textheight]{../figs/adrase.png}
\end{center}
\end{center}
\end{block}
\end{frame}

\section{Métodos híbridos}
\label{sec:org05bc343}

\begin{frame}[label={sec:org4940b55}]{Interpolación Espacial}
\begin{block}{\alert{Objetivo}: mejorar la resolución espacial de medidas dispersas}
\begin{itemize}
\item \alert{Inverse Distance Weighting (IDW)}: determinista (los pesos \(w_i\) son una función inversa de la distancia.)

\[
\widehat{G}_d(x_0) = \frac{\sum_{i=1}^N w_i G_{d}(x_i)}{\sum_{i=1}^N w_i}, \quad w_i=\frac  {1}{d(x_0, x_i)^p}
\]

\item \alert{Ordinary Kriging}: modelo determinista para la media (constante) y estocástico para residuos.
\end{itemize}

\[
  \widehat{G}(\mathbf{s}) = \mu_G + \epsilon_G(\mathbf{s})
\]

\begin{itemize}
\item \alert{Kriging with External Drift (KED)}: modelo determinista para la media incorporando información de una variable con alta densidad espacial.
\end{itemize}

\nocite{Journee.Bertrand2010}
\nocite{Antonanzas-Torres.Canizares.ea2013}
\nocite{Bojanowski.Vrieling.ea2013}
\end{block}
\end{frame}


\begin{frame}[label={sec:orga117031}]{Corrección por topografía}
\begin{center}
\begin{center}
\includegraphics[width=0.9\textwidth]{../figs/downscaling.pdf}
\end{center}
\end{center}

\begin{description}
\item[{Sky-View Factor (SVF)}] Proporción de cielo visible para un receptor horizontal (afecta a la radiación difusa isotrópica)

\item[{Horizon blocking}] Bloqueo de región circunsolar por horizonte: afecta a radiación directa y difusa anisotrópica
\end{description}


\nocite{Bosch.Batlles.ea2010}
\nocite{Tovar-Pescador.Pozo-Vazquez.ea2006}
\nocite{Antonanzas-Torres.MartinezdePison.ea2013}
\nocite{Hofierka.Suri2002}
\end{frame}

\begin{frame}[label={sec:orgd1296b1},fragile]{PVGIS - \texttt{r.sun}}
 \begin{block}{\url{http://re.jrc.ec.europa.eu/pvgis/apps4/pvest.php}}
\begin{itemize}
\item Datos de radiación en el plano horizontal de CM-SAF
\item Permite incorporar la corrección por topografía (SVF y horizon blocking) con perfil estándar o con datos importados.
\end{itemize}
\end{block}
\end{frame}
\end{document}