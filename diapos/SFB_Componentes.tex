% Created 2015-02-07 sáb 16:05
\documentclass[xcolor={usenames,svgnames,dvipsnames}]{beamer}
\usepackage[utf8]{inputenc}
\usepackage[T1]{fontenc}
\usepackage{fixltx2e}
\usepackage{graphicx}
\usepackage{longtable}
\usepackage{float}
\usepackage{wrapfig}
\usepackage{rotating}
\usepackage[normalem]{ulem}
\usepackage{amsmath}
\usepackage{textcomp}
\usepackage{marvosym}
\usepackage{wasysym}
\usepackage{amssymb}
\usepackage{hyperref}
\tolerance=1000
\usepackage{color}
\usepackage{listings}
\usepackage{mathpazo}
\usepackage{gensymb}
\usepackage{amsmath}
\bibliographystyle{plain}
\AtBeginSubsection[]{\begin{frame}[plain]\tableofcontents[currentsubsection,sectionstyle=show/shaded,subsectionstyle=show/shaded/hide]\end{frame}}
\AtBeginSection[]{\begin{frame}[plain]\tableofcontents[currentsection,hideallsubsections]\end{frame}}
\usepackage[emulate=units]{siunitx}
\sisetup{per=fraction, fraction=nice, decimalsymbol=comma}
\newunit{\wattpeak}{Wp}
\newunit{\watthour}{Wh}
\newunit{\amperehour}{Ah}
\hypersetup{colorlinks=true, linkcolor=Blue, urlcolor=Blue}
\setbeamercolor{alerted text}{fg=red!50!black} \setbeamerfont{alerted text}{series=\bfseries}
\usetheme[hideothersubsections]{Goettingen}
\usecolortheme{rose}
\usefonttheme{serif}
\author{Oscar Perpiñán Lamigueiro \\ \url{http://oscarperpinan.github.io}}
\date{}
\title{SFB: Componentes}
\hypersetup{
  pdfkeywords={},
  pdfsubject={},
  pdfcreator={Emacs 24.4.1 (Org mode 8.2.7c)}}
\begin{document}

\maketitle

\section{Introducción}
\label{sec-1}

\begin{frame}[label=sec-1-0-1]{Agua y ESF}
\begin{itemize}
\item Las \alert{curvas de generación fotovoltaica y de consumo de agua están bien adaptadas}: las épocas de mayor calor y radiación solar son de mayor consumo de agua.

\item Se puede utilizar el \alert{agua como medio de acumulación de energía}, evitando baterías con el consiguiente ahorro de costes, a la vez que aumenta la seguridad, eficiencia y fiabilidad.

\item El bombeo de agua directo fotovoltaico es limpio: \alert{no presenta los riesgos de una contaminación del pozo a causa de posibles derrames de combustible}. Asimismo, se evitan los problemas logísticos de suministro y transporte de carburante.
\end{itemize}
\end{frame}


\begin{frame}[label=sec-1-0-2]{Composición}
\includegraphics[height=\textheight]{../figs/EsquemaBombeo_oscar.pdf}
\end{frame}

\section{Motobombas}
\label{sec-2}

\begin{frame}[label=sec-2-0-1]{Definición}
\begin{itemize}
\item Un \alert{motor eléctrico} es una máquina eléctrica que \alert{transforma energía eléctrica en energía mecánica} por medio de interacciones electromagnéticas.

\item Una \alert{bomba} es una \alert{máquina hidráulica} generadora que \alert{transforma la energía mecánica} con la que es accionada \alert{en energía hidráulica del fluido} (agua). Al incrementar la energía del fluido, se aumenta su presión, su velocidad o su altura, todas ellas relacionadas según el principio de Bernoulli.
\end{itemize}
\end{frame}

\subsection{Motores eléctricos}
\label{sec-2-1}

\begin{frame}[label=sec-2-1-1]{Electromagnetismo}
\begin{itemize}
\item Un \alert{campo magnético} ejerce una \alert{fuerza} sobre una \alert{carga en movimiento}.

\item Una \alert{corriente eléctrica crea un campo magnético} en torno al conductor.

\item Un \alert{conductor por el que circula corriente}, situado \alert{en el seno de un campo magnético}, altera este campo magnético, y \alert{experimenta una fuerza} que lo expulsa para \alert{disminuir la alteración}.
\end{itemize}
\end{frame}

\begin{frame}[label=sec-2-1-2]{Electromagnetismo}
\begin{itemize}
\item Entre los puntos extremos de una \alert{espira} estática atravesada por \alert{campo magnético variable}, aparece una \alert{tensión inducida}.

\item Esta tensión es igual a la \alert{variación} (con signo contrario) \alert{del flujo magnético} que atraviesa la espira.

\item Si la espira se cierra, \alert{circulará una corriente} que, a su vez, creará un campo magnético que contrarrestará la variación de flujo.

\item Al circular corriente, la espira experimentará un \alert{par de giro}.

\begin{itemize}
\item \alert{Resultado aprovechable} del motor en forma de potencia mecánica.

\item Restablecer el equilibrio existente antes, intentando \alert{alinear los ejes magnéticos de inductor e inducido}.
\end{itemize}
\end{itemize}
\end{frame}



\begin{frame}[label=sec-2-1-3]{Estator, rotor, inducido e inductor}
\begin{itemize}
\item El elemento que permanece fijo es el estator y el que realiza el giro es el rotor.

\item Según el tipo de motor, el rotor puede ser el inducido y el estator el inductor o viceversa.
\end{itemize}
\end{frame}


\begin{frame}[label=sec-2-1-4]{Frecuencia eléctrica y velocidad}
$$\begin{aligned}
f_{2} & = & f_{1}-n\cdot p\end{aligned}$$

\begin{itemize}
\item $f_{2}$ es la frecuencia en el inducido; $f_{1}$ es la frecuencia en el inductor; $n$ es la velocidad angular; $p$ es el número de polos.

\item Al utilizar colector de delgas (escobillas) en el inducido, la frecuencia en el circuito exterior ($f_{L}$) es diferente a $f_{2}$.
\end{itemize}
\end{frame}

\begin{frame}[label=sec-2-1-5]{Tipos de motores en ESF}
\begin{block}{Motor DC}
\begin{itemize}
\item $f_{1}=0$; $f_{L}=0$;

\item \alert{Estator-Inductor} alimentado por \alert{corriente DC} (o imanes permanentes).

\item El \alert{colector de delgas} transforma la frecuencia de alimentación (DC) en alterna.

\item \alert{Rotor-Inducido gira sincronizado} con la frecuencia \guillemotleft{}transformada\guillemotright{}.
\end{itemize}
\end{block}
\end{frame}

\begin{frame}[label=sec-2-1-6]{Tipos de motores en ESF}
\begin{block}{Motor DC}
\begin{itemize}
\item Los \alert{motores DC con escobillas están sometidos a desgaste}. Necesitan mantenimiento y por tanto deben evitarse con bombas sumergidas.

\item Existen \alert{motores DC sin escobillas}, donde la conmutación se realiza mediante un \alert{circuito electrónico}.

\item No necesitan inversor, tienen buen rendimiento, pero están indicados para \alert{potencias bajas}.
\end{itemize}
\end{block}
\end{frame}

\begin{frame}[label=sec-2-1-7]{Tipos de motores en ESF}
\begin{block}{Motor asíncrono o de inducción}
\begin{itemize}
\item $f_{1}\neq0$;

\item \alert{Estator-inductor} alimentado por una \alert{corriente trifásica alterna}.  Produce un campo giratorio.

\item \alert{Rotor-inducido} constituido por \alert{espiras cortocircuitadas} (jaula de ardilla).

\item Se produce un par que busca alinear el eje de las espiras con el campo inducido. El rotor se mueve siguiendo al campo giratorio.

\item La \alert{velocidad de giro es inferior a la frecuencia de alimentación} (asíncrono).
\end{itemize}
\end{block}
\end{frame}

\begin{frame}[label=sec-2-1-8]{Tipos de motores en ESF}
\begin{block}{Motor asíncrono o de inducción}
\begin{itemize}
\item $f_{2}=f_{1}-n\cdot p$

\item $T\propto\left(\frac{V}{f}\right)^{2}$, $\phi\propto\frac{V}{f}$

\item Son los más comunes, y más baratos que los DC.

\item Tienen \alert{pares de arranque muy bajos}, adecuados para bombas que requieren bajo par de arranque, como las \alert{centrífugas}.
\end{itemize}
\end{block}
\end{frame}

\subsection{Bombas}
\label{sec-2-2}

\begin{frame}[label=sec-2-2-1]{Ecuación de Bernouilli}
\begin{itemize}
\item Conservación de energía
\end{itemize}

$$\frac{\Delta p}{\rho}+\frac{\Delta v^2}{2}+g\cdot\Delta h=cte.$$
\end{frame}

\begin{frame}[label=sec-2-2-2]{Bombas de desplazamiento positivo}
\begin{itemize}
\item \alert{Principio}: cambio de presión

\item El aumento de presión se realiza por el empuje de las paredes de las cámaras que varían su volumen.

\begin{itemize}
\item \alert{Bombas de émbolo alternativo}, en las que existe uno o varios compartimentos fijos, pero de volumen variable, por la acción de un émbolo o de una membrana (bombas de pistones)

\item \alert{Bombas volumétricas}, en las que una masa fluida es confinada en uno o varios compartimentos que se desplazan desde la zona de entrada (de baja presión) hasta la zona de salida (de alta presión) de la máquina. (p.ej. bomba de tornillo).
\end{itemize}
\end{itemize}
\end{frame}

\begin{frame}[label=sec-2-2-3]{Bombas de membrana}
\includegraphics[width=.9\linewidth]{../figs/800px-Bomba_diafragma_impulsando.pdf}

\includegraphics[width=.9\linewidth]{../figs/Bomba_diafragma_aspirando.pdf}
\end{frame}

\begin{frame}[label=sec-2-2-4]{Bombas helicoidales}
\includegraphics[width=.9\linewidth]{../figs/bombatornillo.pdf}
\end{frame}

\begin{frame}[label=sec-2-2-5]{Bombas helicoidales y de membrana}
\begin{itemize}
\item Formadas por un \alert{contorno móvil} que obliga al fluido a avanzar por la máquina por \alert{cambios de volumen}.

\item Son apropiadas para \alert{altos incrementos de presión y bajos caudales}.  Necesitan un \alert{elevado par de arranque} (por tanto no pueden ser acopladas directamente al generador).

\item Las bombas de diafragma, más económicas, requieren el \alert{reemplazo de los diafragmas} cada dos o tres años, dependiendo del fabricante.
\end{itemize}
\end{frame}

\begin{frame}[label=sec-2-2-6]{Bombas rotodinámicas}
\begin{itemize}
\item \alert{Principio}: añadir cantidad de movimiento

\item En este tipo de bombas hay uno o varios rodetes con álabes que giran generando un campo de presiones en el fluido.

\begin{itemize}
\item \alert{Radiales o centrífugas}, el fluido entra por el centro del rodete, que dispone de unos álabes para conducir el fluido, y por efecto de la fuerza centrífuga es impulsado hacia el exterior, donde es recogido por la carcasa o cuerpo de la bomba, que por el contorno su forma lo conduce hacia las tubuladuras de salida o hacia el siguiente rodete (siguiente etapa)
\end{itemize}
\end{itemize}
\end{frame}

\begin{frame}[label=sec-2-2-7]{Bombas centrífugas}
\includegraphics[width=.9\linewidth]{../figs/BombaCentrifuga.pdf}
\end{frame}

\begin{frame}[label=sec-2-2-8]{Bombas centrífugas}
\begin{itemize}
\item Están diseñadas para vencer una \alert{presión más o menos constante}, proporcionando \alert{elevados caudales para bajas alturas manométricas}.

\item Se puede aumentar la altura que son capaces de vencer añadiendo etapas en serie en la misma bomba.

\item Son \alert{bombas simples y robustas, de bajo coste}.

\item Funcionan bien con pequeños pares de arranque.
\end{itemize}
\end{frame}


\begin{frame}[label=sec-2-2-9]{Según la disposición}
\begin{itemize}
\item \alert{Bombas sumergibles}

\begin{itemize}
\item Pozos profundos de pequeño diámetro

\item Normalmente ensambladas con el motor.
\end{itemize}

\item \alert{Bombas flotantes}

\begin{itemize}
\item Instalación en ríos, lagos o pozos de gran diámetro en flotación.

\item Mucho caudal pero poca altura manométrica
\end{itemize}

\item \alert{Bombas de superficie}

\begin{itemize}
\item Instaladas a nivel de suelo (fácil mantenimiento)

\item Tienen un límite en el nivel de succión (8 metros).

\item Si utilizan agua como lubricante, no deben operar en seco para evitar el sobrecalentamiento.
\end{itemize}
\end{itemize}
\end{frame}

\begin{frame}[label=sec-2-2-10]{Configuraciones típicas}
\begin{itemize}
\item \alert{Sistemas de baja potencia (50 a 400 Wp)}

\begin{itemize}
\item Motor DC accionando una bomba de membrana
\end{itemize}

\item \alert{Sistemas de media potencia (400-1500 Wp)}

\begin{itemize}
\item Bomba sumergible centrífuga multietapa con motor asíncrono

\item Motor DC sin escobillas accionando una bomba helicoidal
\end{itemize}

\item \alert{Potencia superior a 1 kWp}

\begin{itemize}
\item Bomba sumergible centrífuga multietapa con motor asíncrono
\end{itemize}
\end{itemize}
\end{frame}
\begin{frame}[label=sec-2-2-11]{}
\includegraphics[width=.9\linewidth]{../figs/Marruecos4.png}
\end{frame}

\section{Acoplamiento generador-motobomba}
\label{sec-3}

\begin{frame}[label=sec-3-0-1]{Motivo}
\begin{itemize}
\item La \alert{potencia y tensión suministrada por un generador FV varían} con la radiación y la temperatura.

\item Las condiciones de funcionamiento \alert{no se adaptan siempre a todos los requerimientos de la motobomba}.

\item Es necesario adaptar las condiciones de funcionamiento de la motobomba al punto de trabajo del generador FV.

\begin{itemize}
\item \alert{Motor AC: variador de frecuencia}

\item \alert{Motor DC: convertidor DC-DC}
\end{itemize}
\end{itemize}
\end{frame}

\begin{frame}[label=sec-3-0-2]{Convertidor DC-DC}
\begin{itemize}
\item Dispositivo que \alert{transforma corriente continua de una tensión a otra}. Suelen ser reguladores de conmutación, dando a su salida una tensión regulada.

\item Se utiliza para alimentar \alert{motores DC con generador FV}.

\item Normalmente no incorporan buscador de MPP.
\end{itemize}
\end{frame}

\begin{frame}[label=sec-3-0-3]{Variador de frecuencia}
\begin{itemize}
\item El variador de frecuencia \alert{transforma una señal alterna de una frecuencia en otra señal alterna de otra frecuencia}.

\item Está compuesto por un rectificador y un inversor con frecuencia variable.

\item Realiza \alert{continuas variaciones en la tensión de generador para alcanzar un valor de referencia}.

\item Para conseguir igualar la tensión de generador y referencia, varía la frecuencia.

\item El rendimiento de un variador a una tensión cercana a la nominal oscila entre el 94 y el 95\%.

\item No suele realizarse seguimiento del punto de máxima potencia (MPP), ni por temperatura ni por radiación.
\end{itemize}
\end{frame}

\begin{frame}[label=sec-3-0-4]{Protecciones}
\begin{block}{Pozo vacío}
\begin{itemize}
\item \alert{Control de frecuencia de salida del variador}.

\item Cuando el motor trabaja en vacío, la corriente consumida baja. Por tanto, el variador debe subir la frecuencia para alcanzar la tensión de referencia.

\item Si se supera la frecuencia de 55 Hz se para el sistema y se marca un intervalo de espera para permitir que el pozo vuelva a llenarse.
\end{itemize}
\end{block}
\end{frame}

\begin{frame}[label=sec-3-0-5]{Protecciones}
\begin{block}{Deposito lleno}
\begin{itemize}
\item \alert{Presostáto en la tubería combinado con una boya en el depósito}.

\item Cuando en el depósito se alcanza un nivel determinado, la boya acciona el cierre de la entrada al depósito.

\item Sin embargo, la bomba sigue elevando agua. De esta forma, la presión dentro de la tubería aumenta hasta accionar el presostato.

\item Se pone en marcha un temporizador para permitir que baje el nivel del depósito.
\end{itemize}
\end{block}
\end{frame}



\begin{frame}[label=sec-3-0-6]{}
\includegraphics[height=0.9\textheight]{../figs/VariadorFrecuencia.jpg}
\end{frame}

\section{Circuito Hidráulico}
\label{sec-4}

\begin{frame}[label=sec-4-0-1]{Circuito hidráulico}
\begin{itemize}
\item Tubería de impulsión

\item Boca de pozo

\item Tubería de distribución y valvulería

\item Depósito
\end{itemize}
\end{frame}

\begin{frame}[label=sec-4-0-2]{}
\includegraphics[width=.9\linewidth]{../figs/CircuitoHidraulico.jpg}
\end{frame}


\begin{frame}[label=sec-4-0-3]{Tubería de Impulsión}
\begin{itemize}
\item Es la tubería instalada a la salida de la bomba.

\item Polietileno de alta densidad y calidad alimentaria

\begin{itemize}
\item Coste menor

\item Tendencia a enrollarse.
\end{itemize}

\item Tuberías autoportantes flexibles

\begin{itemize}
\item Coste mayor

\item Requiere terminales específicos fabricados en acero inoxidable
\end{itemize}
\end{itemize}
\end{frame}

\begin{frame}[label=sec-4-0-4]{}
\includegraphics[width=.9\linewidth]{../figs/Marruecos4.png}
\end{frame}


\begin{frame}[label=sec-4-0-5]{Depósito elevado}
\begin{itemize}
\item \alert{Tamaño adecuado para 1 o 2 días de consumo}
\item Para depósitos pequeños (20 a 1.000 l) debe elegirse un \alert{depósito plástico de color negro}  para evitar aparición de algas y otros contaminantes.
\item El plástico puede ser polietileno de alta densidad para uso alimentario.
\end{itemize}
\end{frame}

\begin{frame}[label=sec-4-0-6]{}
\includegraphics[width=.9\linewidth]{../figs/Bombeo.jpg}
\end{frame}
% Emacs 24.4.1 (Org mode 8.2.7c)
\end{document}