\documentclass{standalone}

\usepackage{circuitikz}

\begin{document}

\begin{circuitikz}[straight voltages]


  \draw
  (0, 0) node[above left] {$-$}to [fuse, o-] ++(2,0) to [short, -*] ++(1,0) 
  (0, -.5) to [fuse, o-] ++(2,0) to [short, -*] ++(1, 0)
  (0, -1)  to [fuse, o-] ++(2,0) to [short, -*] ++(1, 0) coordinate (A);
  \draw
  (0, 5)  to [fuse, o-] ++(2,0) to [short, -*] ++(1, 0) coordinate (B)
  (0, 4.5) to [fuse, o-] ++(2,0) to [short, -*] ++(1, 0)
  (0, 4) node[below left] {$+$} to [fuse, o-] ++(2,0) to [short, -*] ++(1, 0);
  \draw
  (A) to [short] ++(0, 1.5) to [short] ++(2, 0) coordinate (A2) to [short, -o] ++(2, 0) node[below right] {$-$}
  (B) to [short] ++(0, -1.5) to [short] ++(3, 0) coordinate (B2) to [short, -o] ++(1, 0) node[above right] {$+$};
  \draw
  (B2) to [short, *-] ++(0, -4) to [mov, -*] ++(0, -2) coordinate (E) node[ground]{}
  (A2) to [short, *-] ++(0, -1) to [mov, -] ++(0, -1) |- (E);
  \draw
  (A2) to [mov, *-*] (A2|-B2);
\end{circuitikz}


\end{document}


