\documentclass{standalone}

\usepackage{circuitikz}

\begin{document}

\rotatebox{90}
{
\begin{circuitikz}[straight voltages,
  big/.style={circuitikz/blocks/scale=5}]
  \foreach \Y in {-24, 0, 24}
  {  %% Generadores
    \foreach \h in {0, 8, 16}
    {  
      \foreach \i in {0, 1.5, 3}
      {\draw
        (0, \i + \h + \Y) to [pvmodule] ++(3,0)
        to [pvmodule] ++(3,0)
        edge[dashed] ++(2,0)
        ++(2,0) to [pvmodule] ++(3,0)
        to [pvmodule] ++(3,0);
      }
      \draw
      (0,\h + \Y) to [short, *-*] ++(0,1.5)
      to [short, *-*] ++(0,1.5);
      \draw
      (14,\h + \Y) to [short, *-*] ++(0,1.5)
      to [short, *-*] ++(0,1.5);
    }
    %% Cables de salida de cada generador
    \draw[color=red, line width = 1pt] %%negativos
    (0, 0 + \Y) |- ++(16,5) |- ++(4,1)
    (0, 8 + \Y) |- ++(20,-1) 
    (0, 16 + \Y) |- ++(16,-2) |- ++(4,-6);
    \draw[color=blue, line width = 1pt] %%positivos
    (14,0 + \Y) |- ++(3,4) |- ++(3,7)
    (14,11 + \Y) -- ++(3, 0) |- ++(3,1)
    (14,16 + \Y) |- ++(3,-1) |- ++(3,-2);
    %% Caja de paralelos
    \draw
    (20,5.5 + \Y) rectangle node{CP} (24, 13.5 + \Y);
  }  %% Inversor
  \draw
  (32,10) node[sdcacshape, big](A){};
  \draw[color = blue, line width = 2pt] (24,-13) -| ++(3,1) |- ($(A.dc up in) + (0,-0.3)$);
  \draw[color = blue, line width = 2pt] (24,10) -| ++(2,1) |- (A.dc up in);
  \draw[color = blue, line width = 2pt] (24,35) -| ++(3.5,-1) |- ($(A.dc up in) + (0,0.3)$);
  \draw[color = red, line width = 2pt] (24,-15) -| ++(2,1) |- ($(A.dc down in) + (0,-0.3)$);
  \draw[color = red, line width = 2pt] (24,8) -| ++(1,1) |- (A.dc down in);
  \draw[color = red, line width = 2pt] (24,33) -| ++(2.5,-1) |- ($(A.dc down in) + (0,0.3)$);
  \draw
  (A.right) to [tmultiwire] ++(4,0);
\end{circuitikz}
}

\end{document}


