
\chapterprecis{\vfill{}}

\rule[.5ex]{\linewidth}{1pt}

Este documento utiliza hipervínculos para permitir la navegación en
sus contenidos y en información disponible en Internet. Los enlaces
internos (capítulos, secciones, figuras, tablas, ecuaciones y
referencias bibliográficas) están marcados en color rojo. Los enlaces
a documentos externos están marcados en color azul.

\rule[.5ex]{\linewidth}{1pt}
Las simulaciones y cálculos numéricos han sido realizados con el paquete
\texttt{solaR} \citep{Perpinan2012b} integrado en el software libre
\texttt{R-project} \citep{RDevelopmentCoreTeam2013}. Las gráficas
correspondientes a estos cálculos han sido generados con el paquete
\texttt{lattice} \citep{Sarkar2010} de \texttt{R-project}. 

La escritura y edición del documento ha sido realizada el sistema de preparación
de documentos \href{http://www.latex-project.org/}{\LaTeX{}} , empleando
la clase \href{http://www.ctan.org/tex-archive/macros/latex/contrib/memoir/}{\texttt{Memoir}}
y la fuente \href{http://www.tug.dk/FontCatalogue/palatino/}{\texttt{URW Palladio}}.

Las figuras incluidas en el capítulo de Geometría Solar y en el
apartado de Sombras Mutuas en Sistemas de Seguimiento Solar han sido
generadas mediante código de
\href{http://www.tug.org/PSTricks}{\texttt{PSTricks}}
(\texttt{pst-3dplot}, \texttt{pst-V3D}, \texttt{pst-vue3d},
\texttt{pst-grad}, \texttt{pstricks-add}).  Las figuras que recogen
esquemas eléctricos han sido realizadas mediante el conjunto de macros
\href{http://ece.uwaterloo.ca/~aplevich/Circuit_macros/}{\texttt{Circuit
    Macros}} para \LaTeX{}.

El comportamiento de las asociaciones de dispositivos fotovoltaicos
ha sido modelado mediante el software libre de análisis de circuitos
\href{http://www.gnu.org/software/gnucap/}{\texttt{GnuCap}}.

La imagen del sol que adorna la portada ha sido obtenida de
\href{http://www.openclipart.org/}{Open Clipart}, y está disponible en \url{http://www.openclipart.org/detail/553}.



%%% Local Variables:
%%% mode: LaTex
%%% TeX-master: "ESF.tex"
%%% End: 