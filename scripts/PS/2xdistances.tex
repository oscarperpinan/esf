 % Sombras en un cojunto de seguidores  con PSTricks
 % Copyright (C) 2010 Oscar Perpiñán Lamigueiro
 %
 % This program is free software; you can redistribute it and/or
 % modify it under the terms of the GNU General Public License
 % as published by the Free Software Foundation; either version 2
 % of the License, or (at your option) any later version.
 %
 % This program is distributed in the hope that it will be useful,
 % but WITHOUT ANY WARRANTY; without even the implied warranty of
 % MERCHANTABILITY or FITNESS FOR A PARTICULAR PURPOSE.  See the
 % GNU General Public License for more details.
 %
 % You should have received a copy of the GNU General Public License
 % along with this program; if not, write to the Free Software
 % Foundation, Inc., 59 Temple Place - Suite 330, Boston, MA  02111-1307, USA.
\documentclass{article}
\usepackage{auto-pst-pdf}
\usepackage{pst-3dplot, pst-V3D, pst-vue3d, pst-grad, pst-tree}
\usepackage{amsmath}% para pintar funciones con pst-plot
\usepackage{siunitx}
\usepackage{multido}
\usepackage{mathpazo}
\usepackage{pstricks-add}
\begin{document}
\def\az{30 }%signo contrario a las ecuaciones!
\def\beta{45 }
\def\rad{10 }
\def\rang{\rad 2 div }%fundamental dejar el espacio detrás si luego va a formar parte de otra ecuación
\def\rtvec{\rad 1.05 mul }
\def\rtang{\rang 1.2 mul }


%\begin{center}
\begin{pspicture}(-2,0)(11,6)
%\psframe(-2,0)(11,6)
        %\psgrid
  
\psset{unit=6pt} 
\rput{\az}(0,0){\psframe[fillstyle=solid,fillcolor=blue!30](-5,-2.5)(5,2.5)} 
\rput(0,0){$T_X$}
  \rput{\az}(0,-15){\psframe[fillstyle=solid,fillcolor=blue!30](-5,-2.5)(5,2.5)}
\rput(0,-15){$T_0$}

% \psset{linewidth=0.5pt, linestyle=dashed}
% \rput{\az}(0, 0){\psline(5, -2.5)(5, -12.5)}
% \rput{\az}(0, 0){\psline(-5, 2.5)(-10, 2.5)}        


\rput{\az}(0,0){\pnode(5, 2.5){A}}
\rput{\az}(0,0){\pnode(7, 2.5){A2}}
\rput{\az}(0,0){\pnode(8, 2.5){B}}
\psline[linestyle=dashed, linewidth=0.5pt](A)(B)
\rput{\az}(0,-15){\pnode(5, 2.5){C}}
\rput{\az}(0,-15){\pnode(7, 2.5){C2}}
\rput{\az}(0,-15){\pnode(15, 2.5){D}}
\psline[linestyle=dashed, linewidth=0.5pt](C)(D)
\rput{\az}(0,0){\pnode(7, -15){E}}

\psIntersectionPoint(A2)(E)(C2)(D){IP}
\psline{<->}(IP)(A2)
\psRelNode(IP)(A2){.5}{dl}
\uput{2pt}[0]{\az}(dl){$d_L$}


\rput{\az}(0,0){\pnode(5, -2.5){F}}
\rput{\az}(0,0){\pnode(5, -15){G}}
\psline[linestyle=dashed, linewidth=0.5pt](F)(G)
\rput{\az}(0,-15){\pnode(5, 0){H}}
\rput{\az}(0,-15){\pnode(15, 0){I}}
\psIntersectionPoint(F)(G)(H)(I){IP2}
\psline{<->}(IP2)(H)

\psRelNode(IP2)(H){.5}{dw}
\uput{2pt}[90]{\az}(dw){$d_W$}
\end{pspicture}
\end{document}