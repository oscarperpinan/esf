\documentclass{article}
\usepackage{pst-eps, pst-3dplot, pst-V3D, pst-vue3d, pst-grad, pst-tree}
\usepackage{amsmath,pstricks-add}% para pintar funciones con pst-plot
\usepackage{siunitx}
\usepackage{multido}
\usepackage{mathpazo}

\pagestyle{empty}
\begin{document}

	\begin{TeXtoEPS}
\pspicture(-3,-8)(10,1)

\newcommand{\pb}[1]{\psframebox*[fillcolor=blue!30]{\parbox[c]{20ex}{\centering #1}}}



\rput(7, -3){\rnode{CP}{\pb{Caja de Paralelos}}}
\rput(7, -7){\rnode{Inv}{\pb{Inversor}}}
\ncangle[angleA=-90,angleB=90]{CP}{Inv}
\naput[npos=0.2]{$L_{inv}=\SI{85}{\meter}$}

\rput(0,0){\rnode{S1}{\pb{Seguidor 1}}}
\ncangle[angleA=-90,angleB=180]{S1}{CP}
\nbput[npos=0.2]{$L_1=\SI{70}{\meter}$}

\rput(5,0){\rnode{S2}{\pb{Seguidor 2}}}
\ncangle[angleA=-90,angleB=180]{S2}{CP}
\nbput[npos=0.2]{$L_2=\SI{30}{\meter}$}

\rput(0,-2){\rnode{S3}{\pb{Seguidor 3}}}
\ncangle[angleA=-90,angleB=180]{S3}{CP}
\nbput[npos=0.3]{$L_3=\SI{40}{\meter}$}

\rput(5,-2){\rnode{S4}{\pb{Seguidor 4}}}
\ncangle[angleA=-90,angleB=180]{S4}{CP}
\nbput[npos=0.3]{$L_4=\SI{1}{\meter}$}

\endpspicture
	\end{TeXtoEPS}
\end{document} 
